\documentclass[a4paper]{article}
\usepackage{../flpack}

\begin{document}

\providecommand\fname{}
\renewcommand\fname{19-10-04}

\subsection{Kurvintegraler (16.2)}
\begin{defn}[Kurvintegraler]
    Låt \(
        C
    \) vara en kurva med en parametrisering \(
        \ve{r} (t)
    \) där \(
        a \leq t \leq b
    \) och \(
        \ve{r} 
    \) är glatt (smooth), d.v.s.\ att \(
        \ve{r} '(t)
    \) existerar och är kontinuerlig. Låt också \(
        f(x,y)
    \) vara definierad på hela \(
        C
    \).

    \emph{Kurvintegralen} av \(
        f
    \) längs \(
        C
    \) är då arean under grafen \(
        f(x,y)
    \) när \(
        (x,y)
    \) går längs \(
        C
    \).

    Formeln för att räkna ut kurvintegralen av \(
        f
    \) längs \(
        C
    \) (givet att \(
        f
    \) är kontinuerlig på \(
        C
    \)) är \[
        \int_C f \dd s = \int_a^b f(\ve{r} (t)) \abs{\ve{r} '(t)} \dd t.
    \] 

    \begin{proof}[Härledning]
        Härledningen är lik härledningen för formeln för längden av kurvor.
    \end{proof}
\end{defn}

Om \(
    f = 1
\) är ovanstående formeln för längden av en kurva. \(
    \dd s
\) brukar kallas \emph{längdelement}. Formeln brukar förkortas \(
    \dd s = \abs{\ve{r} '(t)} \dd t
\) vilket också är vad som används i formelbladet.

\begin{anm}
    Formeln för kurvintegraler är i termer av en parametrisering \(
        \ve{r} (t)
    \) av en kurva \(
        C
    \) men i själva verket beror inte integralen av parametriseringen 
    utan bara på \(
        C
    \), d.v.s.\ punkterna \(
        \{ \ver(t) \; | \; a \leq t \leq b \}
    \). Man kan därför välja vilken parametrisering som helst av \(
        C
    \).
\end{anm}

\begin{ex}
    Övre halvan av enhetscirkeln kan t.ex.\ parametriseras med \(
        \ver_1 (t) = \lang \cos(t), \sin(t) \rang 
    \), \(
        0 \leq t \leq 2\pi
    \) eller \(
        \ver_2 (t) = \lang t, \sqrt{1-t^2}\rang 
    \), \(
        -1 \leq t \leq 1
    \). Båda funkar för att kurvintegrera över.
\end{ex}

\begin{sats}[Massa och masscentrum för en kurva]
    På liknande sätt som tidigare kan man härleda att om en tråd med densitet
    \(
        \rho(x,y)
    \) går längs en kurva \(
        C
    \) är dess massa \[
        m = \int_C \rho \dd s
    \] och dess masscentrum \[
        (\overline{x}, \overline{y}) = \left( \frac{M_x}{m}, \frac{M_y}{m} \right)
    \] där \(
        M_x = \int_C x \rho(x,y) \dd s
    \) och \(
        M_y = \int_C y \rho(x,y) \dd s
    \).
\end{sats}

\subsection{Kurvintegraler av vektorfält (16.2)}
\begin{defn}[Arbete under konstant kraft]
    I en dimension är \emph{arbetet} utfört av en konstant kraft \(
        F
    \) på en partikel som förflyttar sig sträckan \(
        d
    \) är \[
        W = F \cdot d.
    \] Det kan vara både positivt och negativt beroende på riktningarna på termerna.

    I två/tre dimensioner, om en konstant kraft \(
        \ve{F} 
    \) verkar på en partikel som förflyttas från en punkt \(
        P
    \) till en punkt \(
        Q
    \) blir arbetet som utförs \[
        \ve{F} \sprod \ve{PQ}. 
    \] 
\end{defn}

Anta att vi har en kurva \(
    C
\) som parametriseras av \(
    \ver(t)
\) där \(
    a \leq t \leq b
\) och \(
    \ver
\) är glatt. Om vi har ett kraftfält \(
    \ve{F} (x,y)
\) som varierar vill vi att kurvintegralen av \(
    \ve{F} 
\) längs \(
    C
\) ska ge arbetet som \(
    \ve{F} 
\) utför på en partikel som rör sig längs \(
    C
\) från \(
    \ver(a)
\) till \(
    \ver(b)
\). Med Riemannsummor kan man härleda följande formel, se kapitel 16.2 i boken 
för härledning.

\begin{defn}[Kurvintegral av ett vektorfält]
    \emph{Kurvintegralen} av ett vektorfält \(
        \ve{F} 
    \) längs en kurva \(
        C
    \) som parametriseras av \(
        \ver(t)
    \) där \(
        a \leq t \leq b
    \) är \[
        \int_C \ve{F} \sprod \dd \ver 
            = \int_a^b \ve{F} (\ver(t)) \sprod \ver'(t) \dd t. 
    \] 

    Det brukar förkortas som \(
        \dd \ver(t) = \ver'(t) \dd t
    \) vilket är hur det skrivs på formelbladet.
\end{defn}

\begin{ex}
    Beräkna arbetet som utförs av kraftfältet \(
        \ve{F} = \lang 0, -2y \rang
    \) på en partikel som rör sig längs en kurva \(
        C 
    \) som ges av \(
        \ver(t) = \lang \cos(t), \sin(t)\rang
    \) när \(
        0 \leq t \leq \frac{\pi}{2} 
    \).

    Arbetet ges av \(
        W = \int_C \ve{F} \sprod \dd \ver
    \). Vi börjar med att räkna ut \(
        \ve{F}(\ver(t)) = \ve{F}(\cos t, \sin t) = \lang 0, -2\sin t\rang 
    \) och \(
        \ver'(t) = \lang - \sin t, \cos t\rang
    \). Då får vi att \(
        \ve{F}(\ver(t)) \sprod \ver'(t) = 0 \cdot (-\sin t) + (-2 \sin t \cos t)
            = -\sin(2t)
    \).

    Arbetet blir då 
    \begin{align*}
        W &= \int_C \ve{F} \sprod \dd \ver \\
          &= \int_0^{\frac{\pi}{2}} \ve{F}(\ver(t)) \sprod \ver'(t) \dd t \\
          &= \int_0^{\frac{\pi}{2}} -\sin(2t) \dd t \\
          &= \qty[ \frac{\cos(2t)}{2}]_0^{\frac{\pi}{2}} \\
          &= \frac{\cos(\pi) - \cos(0)}{2} \\
          &= -1. 
    \end{align*}
\end{ex}

\subsection{Huvudsatsen för kurvintegraler (16.3)}
\begin{sats}[Huvudsatsen för kurvintegraler]
    Anta att \(
        C
    \) är en kurva som ges av \(
        \ver(t)
    \) när \(
        a \leq t \leq b
    \) och att \(
        f
    \) är en funktion så att \(
        \grad f 
    \) är kontinuerlig längs \(
        C
    \). 
    
    Då är \[
        \int_C \grad f \sprod \dd \ver = f(\ver(b)) - f(\ver(a)).
    \] 

    Om \(
        \ver(t) = \lang t, 0\rang
    \) och om \(
        f
    \) bara beror på \(
        x
    \) blir detta integralkalkylens fundamentalsats.

    \begin{proof}
        Satsen i allmännhet följer av specialfallet ovan och av kedjeregeln. 
        Beviset finns i kapitel 16.3 i boken.
    \end{proof}
\end{sats}

\begin{defn}[Konservativa vektorfält och potential]
    Ett vektorfält \(
        \ve{F}
    \) är \emph{konservativt} om \(
        \ve{F} = \grad f
    \) för någon funktion \(
        f
    \). \(
        f
    \) kallas för en \emph{potential} till \(
        \ve{F}
    \).
\end{defn}

Konservativa vektorfält är alltså lätta att integrera. Men hur avgör man om ett
vektorfält är konservativt? 

\begin{sats}[Villkor för konservativa vektorfält]
    Om vi antar att vi har ett konservativt vektorfält \(
        \ve{F} = \lang P, Q\rang = \grad f = \lang f_x, f_y\rang
    \) och att \(
        f
    \) har kontinuerliga partiella andraderivator så är \(
        \pdv{P}{y} = \pdv{y}(f_x) = f_{xy} = f_{yx} = \pdv{x}(f_y) = \pdv{Q}{x}
    \).

    Alltså måste \[
        \pdv{Q}{x} = \pdv{P}{y}
    \] om \(
        \ve{F}
    \) är konservativt.
\end{sats}

Vi ska nu se att villkoret ibland medför att \(
    \ve{F}
\) är konservativt. 

\begin{defn}[Sluten kurva]
    En \emph{sluten} (closed) kurva börjar och slutar i samma punkt.
\end{defn}

\begin{defn}[Enkel kurva]
    En \emph{enkel} kurva (simple) korsar aldrig sig själv.
\end{defn}

\begin{defn}[Enkelt sammanhängande mängd]
    En mängd \(
        D
    \) är \emph{enkelt sammanhängande} om den 
    \enquote{inte innehåller några hål}. Mer precist säger vi att för alla 
    enkla slutna kurvor \(
        C
    \) i \(
        D \subseteq \realn^2
    \) ligger alla punkter innanför \(
        C
    \) också i \(
        D
    \).
\end{defn}



\end{document} 