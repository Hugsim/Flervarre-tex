\documentclass[a4paper]{article}
\usepackage{../flpack}

\begin{document}

\providecommand\fname{}
\renewcommand\fname{19-10-18}

\subsection{Rotation av vektorfält (16.5)}
\begin{defn}[Rotation (curl) av vektorfält]
    \emph{Rotationen} av ett vektorfält \(
        \vef = \lang P,Q,R\rang 
    \) i \(
        \realn^3
    \) är \[
        \text{rot} \vef = \curl{\vef} = \lang \pdv{R}{y} - \pdv{Q}{z}
                    , \pdv{P}{z} - \pdv{R}{x} 
                    , \pdv{Q}{x} - \pdv{P}{y} \rang
    \] 

    Formeln är kryssprodukten mellan \(
        \grad
    \) och \(
        \vef
    \). 
\end{defn}

Rotation används bland annat för att formulera två av Maxwells lagar i 
elektromagnetism. 

\begin{sats}[Rotation av gradient]
    Om \(
        f
    \) är en funktion med kontinuerliga partiella andraderivator blir \(
        \curl(\grad{f}) = 
        \begin{vmatrix}
            \uvec{i} & \uvec{j} & \uvec{k} \\ 
            \pdv{x} & \pdv{y} & \pdv{z} \\ 
            \pdv{f}{x} & \pdv{f}{y} & \pdv{f}{z}
        \end{vmatrix}
        = \lang \pdv{f}{y}{z} - \pdv{f}{z}{y}, \pdv{f}{x}{z} - \pdv{f}{z}{x}, \pdv{f}{x}{y} - \pdv{f}{y}{x} \rang
        = \lang 0,0,0\rang
    \) enligt Clairauts sats.
\end{sats}

\begin{sats}
    Om \(
        \vef
    \) är ett vektorfält på \(
        \realn^3
    \) vars komponenter har kontinuerliga partiella derivator och \(
        \curl{\vef} = \lang 0,0,0\rang
    \) så är \(
        \vef
    \) konservativt, d.v.s.\ \(
        \vef = \grad{f}
    \) för någon funktion \(
        f
    \). 

    Detta gäller mer allmänt på områden i \(
        \realn^3
    \) som är enkelt sammanhängande, d.v.s.\ \enquote{utan hål}. Det är mer 
    invecklat att definiera i tre dimensioner än två, och vi gör det inte här.
\end{sats}

Vi kan använda en liknande metod som i \(
    \realn^2
\) för att hitta en potential.

\begin{ex}
    Låt \(
        \vef = \lang z,2y,x+2z\rang
    \) vara ett konservativt vektorfält. Bestäm en potential till \(
        \vef
    \).

    Vi vill alltså lösa \(
        \left\{\begin{matrix}
            f_x = z & \text{(i)} \\ 
            f_y = 2y & \text{(ii)}\\ 
            f_z = x+2z & \text{(iii)}
        \end{matrix}\right.
    \) 
    
    En lösning till (i) är \(
        f_0 = xz
    \) och utifrån det är den allmänna lösningen \(
        f = xz + g(y,z)
    \) (*). Vi sätter in den i (ii) och får \(
        2y = f_y = \{\text{enligt (*)} \} = \pdv{y} (xz + g(y,z)) = g_y(y,z)
    \). En lösning för \(
        g_0
    \) är \(
        g_0 = y^z
    \) och därmed är den allmänna lösningen \(
        g(y,z) = y^2 + h(z)
    \). Om vi stoppar in den i (*) fås \(
        f\xyz = xz + g(y,z) = xz + y^2 + h(z)
    \) (**). Om vi sätter in det i (iii) får vi \(
        x+2z = f_z = \pdv{z}(xz+y^2+h(z))
    \) och därmed \(
        x + 2z = x + h'(z) 
            \implies h'(z) = 2z
            \implies h(z) = z^2 + C
    \). Då får vi genom att gå tillbaka till (**) att \(
        f\xyz = xz+g(yz) = xz + y^2 + h(z) = xz+y^2+z^2 + C
    \) vilket är en potential till \(
        \vef
    \), vilket man vanligtvis bör kontrollera. 
\end{ex}

\subsection{Stokes sats}
\begin{defn}[Positivt orienterad rand till en yta i \(\realn^3\)]
    Låt \(
        S 
    \) vara en yta i \(
        \realn^3
    \) som parametriseras av \(
        \ver(u,v)
    \) där \(
        (u,v) \in D
    \). Randen \(
        \partial S
    \) till ytan \(
        S
    \) är bilden av randen \(
        \partial D
    \) till \(
        D
    \).

    En orientering \(
        \ven
    \) av \(
        S
    \) ger en orientering av \(
        \partial S
    \) genom att man går moturs längs \(
        \partial S
    \) runt \(
        \ven
    \). Om \(
        \partial S
    \) har den orienteringen så säger man att den är \emph{positivt orienterad}.
\end{defn}

\begin{sats}[Stokes sats]
    Låt \(
        S
    \) vara en orienterad yta i \(
        \realn^3
    \) och anta att dess rand \(
        C = \partial S
    \) är en enkel, sluten och positivt orienterad kurva. Låt också \(
        \vef
    \) vara ett vektorfält med kontinuerliga partiella derivator. 
    
    Då gäller att \[
        \int_C \vef \sprod \dd \ver = \iint_S (\curl{F}) \sprod \dd \ve{S}.
    \] 
\end{sats}

Om ett område \(
    D \subseteq \realn^2
\) kan det också ses som en yta \(
    S = \{ (x,y,0) \; | \; (x,y) \in D \}
\) i \(
    \realn^3
\). I detta fallet är Greens formel för \(
    D
\) samma som Stokes sats för \(
    S
\). Mer om det i anteckningarna på kurshemsidan. 

\begin{anm}
    Man ser på liknande sätt som att \(
        \curl(\grad{\vef}) = 0
    \) också att \(
        \div(\curl{\vef}) = 
    \), d.v.s.\ att \(
        \ve{G} = \curl{\vef}
    \) är källfritt.

    Omvänt gäller på \(
        \realn^3
    \) eller andra \enquote{lämpliga områden} \(
        E \subseteq \realn^3
    \) att om \(
        \ve{G}
    \) är källfritt, d.v.s.\ om \(
        \div{\ve{G}} = 0
    \) så är \(
        \ve{G} = \curl{\vef}
    \) för något vektorfält \(
        \vef
    \). 
    
    Inom framför allt fysik är det ett vanligt krav/antagande att  
    vektorfält är källfria, vilket ger relevanta exempel när Stokes sats
    är tillämpbar och användbar. 
\end{anm}

\begin{anm}
    Stokes sats kan också användas för att visa satsen om att vektorfält
    på \(
        \realn^3
    \) är konservativa precis när dess rotation är \(
        0
    \). Beviset finns i avsnitt 16.8 i boken.
\end{anm}

\end{document}
