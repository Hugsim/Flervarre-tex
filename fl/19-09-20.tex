\documentclass[a4paper]{article}
\usepackage{../flpack}

\begin{document}

\providecommand\fname{}
\renewcommand\fname{19-09-20}

\subsection{Längd av kurvor (10.2, 13.3)}
\begin{sats}[Längd av en kurva]
    Låt \(
        \ve{r} (t), a \leq t \leq b
    \) vara en kurva. Då är dess längd \(
        l = \int_a^b \abs{\ve{r}'(t)}\dd t
    \).

    \begin{proof}
        Finns en skiss till ett argument i 10.2 i boken.
    \end{proof}
\end{sats}

\begin{ex}
    Beräkna längden av kurvan med parametrisering \(
        \ve{r} (t) = \lang x(t), y(t) \rang = \lang t^2, t^3\rang
    \) för \(
        0 \leq t \leq 1
    \).

    Då är \(
        \abs{\ve{r}' (t)} 
        = \sqrt{{x'(t)}^2+{y'(t)}^2} 
        = \sqrt{ {(2t)}^2 + {(3t^2)}^2 }
        = \sqrt{ 4t^2+9t^4 }
        = \sqrt{ t^2(4+9t^2) }
        = \sqrt{t^2} \sqrt{4+9t^2}
        = t \sqrt{4+9t^2}
    \).

    Från formeln ovan är \(
        l = \int_0^1 t \sqrt{4+9t^2} \dd t 
        = \left\{\begin{matrix}
            s = 4+9t^2\\ 
            \dd s = 18t \dd t \implies t \dd t = \frac{\dd s}{18} \\ 
            t = 0 \implies s = 4\\ 
            t = 1 \implies s = 13
            \end{matrix}\right.
        = \inv{18} \int_4^{13} \sqrt{s} \dd s
        = \inv{18} \left[ \frac{s^{\frac{3}{2}}}{\frac{3}{2}} \right]_4^{13} 
        = \inv{18} \left( \frac{{13}^{\frac{3}{2}}}{\frac{3}{2}} - \frac{4^{\frac{3}{2}}}{\frac{3}{2}}\right)
        = \inv{27} \left( {13}^{\frac{3}{2}} - 4^{\frac{3}{2}}\right)
    \) 
\end{ex}

\begin{sats}[Längden av en graf]
    Längden av en graf \(
        y = f(x)
    \) är \(
        l = \int_a^b \sqrt{1 + {f'(x)}^2} \dd x
    \). 
    \begin{proof}
        \(
            f(x)
        \)  kan parametriseras med ekvationen \(
            \ve{r} (t) = \lang t, f(t) \rang 
        \). Då blir \(
            \abs{\ve{r} '(t)} = \sqrt{1 + {f'(x)}^2}
        \) och längden därmed \(
            l = \int_a^b \sqrt{1 + {f'(x)}^2} \dd x
        \).

        Alternativ härledning finns i föreläsningsslides.
    \end{proof}
\end{sats}

\subsection{Rörelse av partiklar i rummet (13.4)}
Man kan använda en vektorvärd funktion \(
    \ve{r} (t)
\) för att beskriva hur en partikel rör sig genom rummet. Vid tiden \(
    t
\) befinner sig partikeln i \(
    \ve{r} (t)
\). 

\begin{defn}[Hastighetsvektor och fart]
    \emph{Hastighetsvektorn} \(
        \ve{v} (t)
    \) för en sådan partikel ovan är \(
        \ve{r} '(t)
    \) och farten är \(
        \abs{\ve{v} (t)}
    \). 
\end{defn}

\begin{defn}[Accelerationsvektor]
    \emph{Accelerationsvektorn} för en partikel som ovan är \(
        \ve{a} (t) = \ve{r} ''(t)
    \).
\end{defn}

Dessa definitioner används för att beskriva olika fysiska fenomen. Ett exempel
på en sådan är Newtons andra lag:

\begin{sats}[Newtons andra lag]
    Om kraften \(
        \ve{F} (t) 
    \) verkar på en partikel med massa \(
        m
    \) så är \(
        \ve{F} (t) = m \cdot \ve{a} (t)
    \).
\end{sats}

\begin{ex}
    Anta att bara tyngdkraften verkar på en partikel med massa \(
        m
    \), d.v.s.\ \(
        \ve{F} = \lang 0, 0, -mg\rang  
    \). Anta också att partikeln vid \(
        t = 0
    \) har position \(
        \ve{r} (0) = \lang a,b,c\rang
    \) och hastighet \(
        \ve{v} (0) = \lang d, e, f\rang
    \). Verifiera att partikeln med position \(
        \ve{r}(t)  = \lang a+dt, b+et, c+ft-\frac{gt^2}{2}\rang
    \) uppfyller dessa villkor och rör sig enligt Newtons andra lag.

    Vi kontrollerar detta: Vi stoppar in \(
        t = 0
    \) i \(
        \ve{r} 
    \) och får \(
        \ve{r} (0) = \lang a, b, c\rang
    \) vilket stämmer. Vi deriverar nu för att få \(
        \ve{v} 
    \) och får \(
        \ve{v} (t) = \ve{r} '(t) = \lang d, e, f-gt\rang
    \). Kontrollerar genom att stoppa in \(
        t = 0
    \): \(
        \ve{v} (0) = \lang d, e, f \rang
    \). Kontrollerar nu med ytterligare en derivata \(
        \ve{a} (t) = \ve{v} '(t) = \lang 0, 0, -g\rang
    \). Använder Newtons andra lag för att kontrollera \(
        m \cdot \ve{a} (t) = \lang 0, 0, -mg\rang = \ve{F} 
    \).
\end{ex}

\dquote{ \(0\) är ju en typisk konstant.}{Mr.\ Väsentligen}

\dquote{Du har ju nog rätt idé, men vad menar du?}{Mr.\ Väsentligen}

\dquote{Ja, det hade man kunnat tänka sig, men så är det inte.}{Mr.\ Väsentligen}

\end{document} 