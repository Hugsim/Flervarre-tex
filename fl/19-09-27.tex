\documentclass[a4paper]{article}
\usepackage{../flpack}

\begin{document}

\providecommand\fname{}
\renewcommand\fname{19-09-27}

\subsection{Integraler över allmänna områden (forts.)}
\begin{sats}[Volym mellan grafer]
    Om \(
        f(x,y) \geq g(x,y) 
    \) ges volymen av området mellan \(
        z = g(x,y)
    \) och \(
        z = f(x,y)
    \) där \(
        (x,y) \in D
    \) ges av \(
        \iint_D f(x,y) - g(x,y) \dd A 
    \).
\end{sats}

\subsection{Dubbelintegraler och polära koordinater (15.3)}
\begin{ex}
    Beräkna volymen av området mellan konen \(
        x^2+y^2 = z^2, z \geq 0
    \) och planet \(
        z = 1
    \).

    Den första ekvationen ger en kon eftersom \(
        z = \sqrt{x^2+y^2}
    \) där \(
        z
    \) är avståndet från \(
        (x,y)
    \) till origo. 

    Det beror inte på vinkeln \(
        \theta
    \) i \(
        xy
    \)-planet, så vi kan rita \(
        z = r
    \) för ex.\ \(
        \theta = 0
    \), där är \(
        r = x
    \), och sedan roterar vi kring \(
        z
    \)-axeln.

    Ytorna \(
        z = \sqrt{x^2+y^2}
    \) och \(
        z = 1
    \) skär varandra i \(
        \sqrt{x^2+y^2} = 1
    \), d.v.s.\ \(
        x^2+y^2 = 1
    \) och \(
        z = 1
    \). Området mellan de här ytorna ges då av \(
        \sqrt{x^2+y^2} \leq z \leq 1
    \), \(
        (x,y) \in D = \{(x,y)\; |\; x^2+y^2 \leq 1\}
    \). \(
        D
    \) sett som ett typ I-område begränsas av \(
        y = \sqrt{1-x^2}
    \) och \(
        y = -\sqrt{1-x^2}
    \). Alltså är \(
        D = \{(x,y)\; |\; -1 \leq x \leq 1,\; -\sqrt{1-x^2} \leq y \leq \sqrt{1-x^2}\}
    \). 

    Områdets volym blir då \(
        \iint_D 1-\sqrt{1-x^2} \dd A 
        = \int_{-1}^1 \int_{-\sqrt{1-x^2}}^{\sqrt{1-x^2}} 1-\sqrt{x^2+y^2} \dd y \dd x
    \). Detta går att beräkna, men blir krångligt.
\end{ex}

Idag ska vi snacka om ett betydligt enklare sätt att räkna ut detta.

\subsubsection{Polära koordinater}
En punkt \(
    (x,y) \in \realn^2
\) kan beskrivas med \emph{polära koordinater} \(
    (r, \theta)
\) genom \(
    \left\{\begin{matrix}
        x = r \cos(\theta) \\ 
        y = r \sin(\theta)
    \end{matrix}\right.
\) där \(
    \theta
\) är vinkeln mot \(
    x
\)-axeln och \(
    r
\) är avståndet från origo.

\begin{ex}
    Området \(
        D  =\{(x,y) \; | \; x^2+y^2 \leq 4,\; y \leq x\}
    \) ser ut som följande:
    \f

    I polära koordinater är \(
        D = \{ (r, \theta) \; | \; 0 \leq r \leq 2, \; -\frac{3\pi}{4}  \leq \theta \leq \frac{\pi}{4}  \}
    \).
\end{ex}

\begin{defn}[Polära rektanglar och deras area]
    Ett område är en \emph{polär rektangel} om det i polära koordinater ges av 
    \(
        a \leq r \leq b
    \) och \(
        \alpha \leq \theta \leq \beta
    \). 

    \f

    Arean för den rektangeln är \(
        \frac{(b^2-a^2)(\beta - \alpha)}{2} 
    \).
\end{defn}

\subsubsection{Integration i polära koordinater}
\begin{sats}[Integration av polära rektanglar]
    Låt \(
        D 
    \) vara en polär rektangel som ges av \(
        a \leq r \leq b, 
        \alpha \leq \theta \leq \beta
    \), och anta att \(
        f 
    \) är kontinuerlig på \(
        D
    \).

    Då är \(
        \iint_D f \dd A = \int_\alpha^\beta \int_a^b f(r\cos(\theta), r\sin(\theta)) \cdot r \; \dd r \dd \theta
        = \int_a^b \int_\alpha^\beta f(r\cos(\theta), r\sin(\theta)) \cdot r \; \dd \theta \dd r
    \).

    \begin{obs}
        Vänsterledet är en integral över ett allmänt område. 
        Högerledet är upprepade integraler.
    \end{obs}

    Man brukar ofta sammanfatta formlerna som \(
        \dd A = r \dd r \dd \theta = r \dd \theta \dd r
    \) eller \(
        \dd x \dd y = r \dd r \dd \theta = r \dd \theta \dd r
    \). Den första sammanfattningen används i formelbladet.

    \begin{proof}
        Idén bakom beviset är samma som för ett \enquote{vanligt} 
        koordinatsystem, men istället för att dela in i rektanglar delar vi 
        in i polära rektanglar istället. Finns ganska komplett i 
        föreläsningsslidesen.
    \end{proof}
\end{sats}

\begin{ex}
    Vi går tillbaka till exemplet i början av föreläsningen för att se 
    om det blir enklare med polära koordinater.

    Beräkna \(
        \iint_D 1-\sqrt{x^2+y^2} \dd A
    \) där \(
        D = \{ (x,y)\;|\; x^2+y^2\leq 1 \}
    \).

    I polära koordinater ges \(
        D
    \) av \(
        0 \leq r \leq 1, 0 \leq \theta \leq 2 \pi
    \). Vår funktion är då
    \begin{align*}
        f(x,y) &= 1 - \sqrt{x^2+y^2} = \sqrt{(r \cos(\theta))^2 + (r \sin(\theta))^2}\\
            &= \sqrt{r^2 (\cos^2(\theta) + \sin^2(\theta))} \\
            &= \sqrt{r^2}\\
            &= r\\
            &= f(r, \theta).
    \end{align*} 
    
    Då blir 
    \begin{align*}
        \iint_D 1-\sqrt{x^2+y^2} \dd A &= \iint_D 1-r \dd A \\
            &= \int_0^1 \int_0^{2\pi} (1-r)r\; \dd \theta \dd r\\
            &= \int_0^1 \left[ (r-r^2) \theta \right]_{\theta=0}^{2\pi} \dd r\\
            &= 2 \pi \int_0^1 r-r^2 \dd r\\
            &= 2 \pi \left[ \frac{r^2}{2} - \frac{r^3}{3} \right]_0^1 \\
            &= 2\pi(\half - \inv{3}) \\
            &= \frac{\pi}{3}.
    \end{align*}
\end{ex}

\end{document} 