\documentclass[a4paper]{article}
\usepackage{../flpack}

\begin{document}

\providecommand\fname{}
\renewcommand\fname{19-09-13}

\subsection{Riktningsderivator och gradienter}
\begin{påm}
    Låt \(
        \ve{u} = \lang u_1, u_2\rang \text{ och } \ve{v} = \lang v_1, v_2\rang
    \) vara vektorer.

    Skalärprodukten är då \[
        \ve{u} \cdot \ve{v} = u_1v_1 + u_2v_2.
    \]

    Längden av \(
        \ve{u} \text{ är } \)
        \[\abs{\ve{u} } = \sqrt{\ve{u} \cdot \ve{u} } = \sqrt{u_1^2 + u_2^2}.\]
    

    Det gäller också att \[
        \ve{u} \cdot \ve{v} = \abs{\ve{u} } \cdot \abs{\ve{v} } \cdot \cos(\theta)
    \] där \(
        \theta
    \) är vinkeln mellan \(
        \ve{u} \text{ och } \ve{v} 
    \).

    Det följer av förra påståeendet att för två vinkelräta vektorer är \(
        \ve{u} \cdot \ve{v} = 0
    \).
\end{påm}

\begin{sats}[Största värde för riktningsderivata]
    \(
        D_{\ve{u}} f(x,y)
    \) är som störst när \(
        \ve{u} 
    \) pekar i samma riktning som \(
        \nabla f(x,y)
    \) d.v.s.\ när \(
        \ve{u} = \bif{\nabla f(x,y)}{\abs{\nabla f(x,y)}} 
    \).

Alltså \(
    f(x,y)
\) växer som mest när \(
    (x,y)
\) rör sig från \(
    (x_0, y_0)
\) i riktningen \(
    \bif{\nabla f(x_0, y_0)}{\abs{\nabla f(x,y)}}
\).

\begin{proof}
    \(
        D_{\ve{u}} f(x,y) = \ve{u} \cdot \nabla f(x,y) 
        = \abs{\ve{u}} \cdot \abs{\nabla f(x,y)} \dot \cos(\theta)
    \) där \(
        \theta
    \) är vinkeln mellan \(
        \ve{u} \text{ och } \nabla f(x,y)
    \). Eftersom \(
        \ve{u}
    \) är en enhetsvektor vet vi att \(
        \abs{\ve{u}} = 1
    \). \(
        \abs{\nabla f(x,y)} \cos(\theta)
    \) blir som störst när \(
        \cos(\theta) = 1
    \), d.v.s. \(
        \theta = 0
    \). Det betyder att \(
        \ve{u}
    \) och \(
        \nabla f(x,y)
    \) pekar i samma riktning.
\end{proof}
\end{sats}

\begin{anm}
    I princip allt från denna veckan generaliserar enkelt från två till 
    tre eller fler variabler.
\end{anm}

\subsection{Tangentplan till nivåytor (14.6)}
En nivåyta bestäms av \(
    F(x,y,x) = k
\) för något konstant \(
    k \in \mathbb{R}
\).

En graf \(
    z = f(x,y)
\) är ett specialfall av en nivåyta, eftersom vi kan ta \(
    F(x,y,z) = z - f(x,y)
\) och \(
    k = 0
\).

Ska definiera tangentplan till allmän nivåyta. Vill definiera tangentplanet till
\(
    F(x,y,z) = k
\) i \(
    (x_0, y_0, z_0)
\) som det plan som genom \(
    (x_0, y_0, z_0)
\) s.a.\ det för varje kurva \(
    \ve{r} (t) = \lang x(t), y(t), z(t)\rang
\) som uppfyller \(
    \ve{r}(t_0) = \lang x_0, y_0, z_0\rang
\) och ligger i nivåytan så innehåller tangentplanet \(
    \ve{r} ' (t_0)
\). 

\f

Anta att \(
    \ve{r} 
\) är en sådan kurva. Då är \(
    \forall t : F(x(t), y(t), z(t)) = k
\). Nu deriverar vi båda sidor med avseende på \(
    t
\) m.h.a.\ kedjeregeln så att vi får 
\begin{align*}
    F_x\cdot x'(t) + F_y \cdot y'(t) + F_z \cdot z'(t) &= 0 \\
    \nabla F(\ve{r}(t) ) \cdot \ve{r} '(t) &= 0.
\end{align*}
Om vi tar \(
    t = t_0
\) får vi att \(
    \nabla F(\ve{r} (t_0)) = \nabla F(x_0, y_0, z_0) 
\) är vinkelrät mot \(
    \ve{r} '(t_0)
\).

\begin{påm}
    En normalvektor \(
        \ve{n} 
    \) till ett plan är en nollskild vektor som är vinkelrät mot alla vektorer 
    i planet. Om planet går genom \(
        (x_0, y_0, z_0)
    \) ges hela planet av \[
        \ve{n} \cdot \lang x-x_0, y-y_0, z-z_0\rang = 0.
    \]
\end{påm}

\begin{defn}[Tangentplan till nivåyta]
    \emph{Tangentplanet} till en nivåyta \(
        F(x,y,z) = k
    \) genom en punkt \(
        (x_0, y_0, z_0)
    \) där \(
        \nabla F(x_0,y_0,z_0) \neq \ve{0} 
    \) är \(
        \nabla F(x_0, y_0, z_0) \cdot \lang x-x_0, y-y_0, z-z_0\rang = 0
    \).
\end{defn}

Det är definierat s.a.\ det uppfyller vad vi ville att det skulle uppfylla.

\begin{obs}
    Om \(
        F(x,y,z) = z-f(x,y)
    \) är nivåytan där \(
        F = 0
    \) samma som grafen \(
        z = f(x,y)
    \). Eftersom \(
        \nabla F = \lang -f_x, -f_y, 1\rang
    \) får i en punkt \(
        (a, b, f(a,b)) 
    \) samma tangentplan \(
        \lang -f_x(a,b), -f_y(a,b), 1\rang \cdot \lang x-a, y-b, z-f(a,b)\rang
    \) oavsett om man ser ytan som en nivåyta eller grafen till en funktion.
\end{obs}

\begin{ex}
    Bestäm tangentplanet till ytan \(
        x^2-y^2+z^2 = 1
    \) i punkten \(
        (1, 1, 1)
    \). 

    Ytan är nivåytan \(
        F(x,y,z) = 1
    \) där \(
        F(x,y,z) = x^2-y^2+z^2
    \).

    \(
        \nabla F = \lang 2x, -2y, 2z\rang 
    \) och \(
        \nabla F(1,1,1) = \lang 2, -2, 2\rang
    \). Då är tangentplanet \(
        \nabla F(1,1,1) \cdot \lang x-1, y-1, z-1 \rang = 0
    \) vilket ger 
    \begin{align*}
        2(x-1)-2(y-1)+2(z-1)&=0\\
        x-1-y+1+z-1 &= 0 \\
        x-y+z &= 1.
    \end{align*}
\end{ex}

\subsection{Lokala extremvärden}
\begin{defn}[Lokalt maximum]
    \(
        f(x,y) 
    \) har ett \emph{lokalt maximum} i \(
        (a,b)
    \) om \(
        f(x,y) \leq f(a,b)
    \) gäller för \(
        (x,y) \text{ nära } (a,b)
    \). (D.v.s.\ det gäller för \(
        (x,y) 
    \) i någon cirkelskiva kring \(
        (a,b)
    \).)
\end{defn}

\begin{defn}[Absolut/Globalt maximum]
    \(
        f(x,y) 
    \) har ett \emph{absolut/globalt maximum} i \(
        (a,b)
    \) om \(
        f(x,y) \leq f(a,b)
    \) gäller för alla \(
        (x,y)
    \) i definitionsmängden till \(
        f
    \). 
\end{defn}

\begin{defn}[Lokalt minimum]
    \(
        f(x,y) 
    \) har ett \emph{lokalt minimum} i \(
        (a,b)
    \) om \(
        f(x,y) \geq f(a,b)
    \) gäller för \(
        (x,y) \text{ nära } (a,b)
    \). (D.v.s.\ det gäller för \(
        (x,y) 
    \) i någon cirkelskiva kring \(
        (a,b)
    \).)
\end{defn}

\begin{defn}[Absolut/Globalt minimum]
    \(
        f(x,y) 
    \) har ett \emph{absolut/globalt minimum} i \(
        (a,b)
    \) om \(
        f(x,y) \geq f(a,b)
    \) gäller för alla \(
        (x,y)
    \) i definitionsmängden till \(
        f
    \). 
\end{defn}

\begin{defn}[Extremvärde]
    Ett \emph{extremvärde} är ett minimum eller maximum.
\end{defn}

\begin{ex}
    \(
        f(x,y) = 3x^2+y^2
    \) har ett lokalt och globalt minimum, var?

    Inses lätt att det sker i \(
        (0,0)
    \) då den annars är strikt positiv för \(
        x, y, \in \mathbb{R}
    \).
\end{ex}

\begin{påm}
    I en variabel, om \(
        f
    \) är deriverbar och har ett (lokalt) extremvärde i \(
        a
    \) är \(
        f'(a) = 0
    \).
\end{påm}

\begin{sats}[Extremvärde i en punkt]
    Om \(
        f(x,y)
    \) har ett (lokalt) extremvärde i punkten \(
        (a,b)
    \) och \(
        f_x(a,b)
    \) och \(
        f_y(a,b)
    \) existerar är \(
        f_x(a,b) = 0
    \) och \(
        f_y(a,b) = 0
    \).

    \begin{proof}
        Om vi låter \(
            h(x) = f(x,b)
        \) ha ett extremvärde i \(
            x=a
        \) gäller \(
            h'(a) = 0
        \) och \(
            h'(a) = f_x(a,b) 
        \), vilket ger att \(
            f_x(a,b) = 0
        \).

        \(
            f_y(a,b) = 0
        \) visas analogt.
    \end{proof}
\end{sats}

\end{document} 