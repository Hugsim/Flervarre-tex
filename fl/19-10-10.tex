\documentclass[a4paper]{article}
\usepackage{../flpack}

\begin{document}

\providecommand\fname{}
\renewcommand\fname{19-10-10}

\subsection{Parametriserade ytor (16.6)}
En kurva \(
    C
\) har dimension \(
    1
\), d.v.s.\ den beskrivs med en parameter \(
    \ver(t) 
\). 

\begin{defn}[Parametriserade ytor]
    En \emph{parametriserad yta} \(
        S
    \) är alla punkter \(
        \{ \ver(u,v) \; | \; (u,v) \in D \subseteq \realn^2 \}
    \) där \(
        \ver : D \to \realn^3
    \) är en vektorvärd funktion. 
\end{defn}

En yta har då dimension \(
    2
\), d.v.s.\ den beskrivs av två parametrar \(
    \ver(u,v)
\). 

Vi skriver ofta parametriseringen i sina komponenter \(
    \ver(u,v) = \lang x(u,v), y(u,v), z(u,v) \rang
\). 

\begin{ex}
    Låt \(
        f : D \to \realn
    \), \(
        D \subseteq \realn^2
    \). 

    Då är grafen \(
        z = f(x,y)
    \) en parametriserad yta med \(
        \ver(u,v) = \lang u, v, f(u,v) \rang
    \). Dock brukar vi ofta istället skriva \(
        \ver(x,y) = \lang x, y, f(x,y) \rang
    \).
\end{ex}

\begin{ex}
    Låt \(
        g : [a,b] \to [0, \infty)
    \). 

    Då ges rotationsytan kring \(
        z
    \)-axeln med radien \(
        g(z)
    \) av \(
        \ver(u,v) = \lang g(u) \cos v, g(u) \sin v, u\rang
    \) där \(
        (u,v) \in [a,b] \times [0, 2\pi]
    \).

    I cylindriska koordinater svarar det mot att \(
        r = g(u) 
    \), \(
        \theta = v
    \) och \(
        z = u
    \).

    Vi kan också skriva \(
        \ver(z, \theta) = \lang g(z) \cos \theta, g(z) \sin \theta, z\rang
    \).
\end{ex}

\subsection{Tangentplan till parametriserade ytor (16.6)}


    Låt \(
        S
    \) vara en yta som parametriseras av \(
        \ver(u,v)
    \). Fixera en punkt \(
        \lang a, b, c\rang = \ver(u_0, v_0)
    \) på \(
        S
    \). Låt också \(
        \ver_u(u_0, v_0)
    \) och \(
        \ver_v(u_0,v_0)
    \) vara de partiella derivatorna av \(
        \ver
    \) med avseende på \(
        u
    \) respektive \(
        v
    \) i \(
        (u_0,v_0)
    \). Kurvorna \(
        \ver_1(u) = \ver(u,v_0) 
    \) och \(
        \ver_2(v) = \ver(u_0, v)
    \) ligger då i \(
        S
    \) och deras tangentvektorer i \(
        u_0
    \) respektive \(
        v_0
    \) är de partiella derivatorna ovan. 

    \f

\begin{defn}[Glatta (smooth) ytor i en punkt]
    En yta \(
        S
    \) är \emph{glatt} i \(
        \ver(u_0, v_0)
    \) om \(
        \ver_u(u_0,v_0) \times \ver_v(u_0,v_0) \neq 0
    \).

    Det är samma sak som att vektorerna inte är parallella. Det är också samma sak 
    som att de spänner upp ett plan.
\end{defn}
\begin{defn}[Tangentplan till en parametriserad yta]
    Om \(
        S
    \) är glatt i en punkt \(
        (a,b,c) = \ver(u_0,v_0)
    \) är dess \emph{tangentplan} det plan som går genom \(
        (a,b,c)
    \) och innehåller alla tangentvektorer till kurvor i \(
        S
    \) som går genom \(
        (a,b,c)
    \).

    Tangentplanet har normalvektor \(
        \ve{n} = \ver_u(u_0,v_0) \times \ver_v(u_0,v_0)
    \) och om man skriver \(
        \ve{n} = \lang n_1, n_2, n_3\rang
    \) så ges tangentplanet av \[
        n_1(x-a) + n_2(y-b) + n_3(z-c) = 0
    \] eller ekvivalent \[
        \ve{n} \sprod \lang x-a, y-b, z-c\rang = 0.
    \]

    Härledningen är lik den för tangentplanet av en nivåyta, den finns också
    på Canvas.
\end{defn}

Om \(
    S
\) är grafen \(
    z = f(x,y)
\) får vi samma formel som tidigare för tangentplanet, eftersom om \(
    \ver(x,y) = \lang x,y,f(x,y)\rang 
\) blir \(
    \ver_x \times \ver_y = 
    \begin{vmatrix}
        \uvec{i} & \uvec{j} & \uvec{k} \\ 
        1 & 0 & f_x \\ 
        0 & 1 & f_y
    \end{vmatrix} = \lang -f_x, -f_y, 1 \rang
\) och \(
    \ver(a,b) = \lang a, b, f(a,b)\rang
\) så att tangentplanet blir \(
    -f_x(a,b) (x-a) - f_y(a,b)(y-b) + 1 \cdot (z - f(a,b)) = 0
\) vilket är samma sak som \(
    z = f(a,b) + f_x(a,b)(x-a)+f_y(a,b)(y-b)
\).

\subsection{Arean av parametriserade ytor (16.6)}
\begin{sats}[Arean för parametriserade ytor]
    Arean för en yta \(
        S
    \) som parametriseras av \(
        \ver(u,v)
    \) där \(
        (u,v) \in D \subseteq \realn^2
    \) är \[
        A = \iint_D \abs{ \ver_u(u,v) \times \ver_v(u,v) } \dd A.
    \] 

    \begin{proof}
        Härledning finns i föreläsningsslides.
    \end{proof}
\end{sats}

\begin{ex}
    Om \(
        S
    \) är grafen till en funktion \(
        f
    \) och \(
        S
    \) parametriseras av \(
        \ver(x,y) = \lang x, y, f(x,y) \rang
    \) där \(
        (x,y) \in D
    \) såg vi att \(
        \ver_x \times \ver_y = \lang -f_x, -f_y, 1\rang
    \). Eftersom \(
        \abs{\ver_x \times \ver_y} = \sqrt{ 1 + f_x^2 + f_y^2 }
    \) ger det formeln från tidigare som säger att arean är \(
        A = \iint_D \sqrt{1 + f_x^2 + f_y^2} \dd A
    \).
\end{ex}

\begin{ex}
    Låt \(
        S
    \) vara rotationsytan \(
        \ver(u,v) = \lang g(u) \cos v, g(u) \sin v, u\rang
    \) där \(
        (u,v) \in D = [a,b] \times [0,2\pi]
    \). 

    Då kan vi räkna fram (finns i föreläsningsanteckningar på Canvas) att arean 
    av \(
        S
    \) är \(
        A = 2\pi \int_a^b g(u) \sqrt{1+g'(u)^2} \dd u
    \).

    Den formeln finns i avsnitt 8.2 där den härleds på ett annat sätt.
\end{ex}

\end{document}
