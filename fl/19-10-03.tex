\documentclass[a4paper]{article}
\usepackage{../flpack}

\begin{document}

\providecommand\fname{}
\renewcommand\fname{19-10-03}

\subsection{Sfäriska koordinater och trippelintegraler (15.8)}
Kom ihåg att en punkt \(
    (x,y)
\) i planet kan beskrivas i polära koordinater med ett avstånd \(
    r = \sqrt{x^2+y^2}
\) från origo och en vinkel \(
    \theta 
\) från positiva \(
    x
\)-axeln. 

\begin{defn}[Sfäriska koordinater]
    En punkt \(
        \xyz
    \) i rummet kan beskrivas med ett avstånd \(
        \rho
    \) till origo och två vinklar, vanligtvis \(
        \theta
    \) till positiva \(
        x
    \)-axeln i \(
        xy
    \)-planet och \(
        \phi
    \), vinkeln till positiva \(
        z
    \)-axeln. \(
        (\rho, \theta, \phi)
    \) kallas \emph{sfäriska koordinater} och ges av \(
        \left\{\begin{matrix}
            x = \rho \sin(\phi)\cos(\theta) \\
            y = \rho \sin(\phi)\sin(\theta) \\
            z = \rho \cos(\phi)
        \end{matrix}\right.
    \). 
    
    Man brukar kräva att \(
        \rho \geq 0
    \), \(
        0 \leq \phi \leq \pi
    \) och oftast antingen \(
        0 \leq \theta \leq 2\pi
    \) eller \(
        -\pi \leq \theta \leq \pi
    \).

    \begin{proof}[Härledning]
        ~\\
        \begin{enumerate}
            \item Vi börjar med punkten \(
                P_0 = (0,0,\rho)
            \) som ligger rakt upp från origo. Den har \(
                \rho 
            \) rätt, \(
                \phi = 0
            \) och \(
                \theta
            \) odefinierat.

            \item Om vi roterar punkten i \(
                xz
            \)-planet får vi \(
                P_1 = (\rho \sin(\phi), 0, \rho \cos(\phi))
            \). Den har \(
                \rho 
            \) rätt, \(
                \phi
            \) rätt och \(
                \theta = 0
            \) eftersom den ligger rakt över \(
                x
            \)-axeln.

            \item Om vi roterar i \(
                xy
            \)-planet får vi \(
                (\rho \sin(\phi)\cos(\theta), \rho \sin(\phi)\sin(\theta), \rho \cos(\phi))
            \).
        \end{enumerate}
    \end{proof}
\end{defn}

\begin{sats}[Formel för integration i sfäriska koordinater]
    Om \(
        E 
    \) är en \enquote{sfärisk låda}, d.v.s.\ att \(
        a \leq \rho \leq b
    \), \(
        \alpha \leq \theta \leq \beta
    \) och \(
        c \leq \phi \leq d
    \) är \[
        \iiint_E f\xyz \dd V = \int_\alpha^\beta \int_a^b \int_c^d f(\rho \sin(\phi)\cos(\theta), \rho \sin(\phi)\sin(\theta), \rho \cos(\phi)) \rho^2 \sin(\phi) \; \dd \phi \dd \rho \dd \theta.
    \] 

    Det brukar sammanfattas som \[
        \dd V = \rho^2 \sin(\phi) \; \dd \phi \dd \rho \dd \theta.
    \] 

    \begin{proof}
        Kort idé till varför finns i avsnitt 15.8 i boken.
    \end{proof}
\end{sats}

\begin{ex}
    Beräkna \(
        \iiint_E {(x^2+y^2+z^2)}^2 \dd V
    \) där \(
        E
    \) är området som ligger ovanför konen \(
        z = \sqrt{x^2+y^2}
    \) och innanför sfären \(
        x^2+y^2+z^2 = 4
    \).

    \f

    Vinkeln \(
        \phi
    \) mot positiva \(
        z
    \)-axeln är \(
        \frac{\pi}{4} 
    \) eftersom om man bara kollar i \(
        xz
    \)-planet beskrivs linjen som \(
        z = x
    \) vilken har vinkeln \(
        \frac{\pi}{4} 
    \) mot \(
        z
    \)-axeln. Området ovanför konen ges då av att \(
        0 \leq \phi \leq \frac{\pi}{4} 
    \). Sfären \(
        x^2+y^2+z^2 = 4
    \) ges av \(
        \rho = 2
    \), då är området innanför sfären \(
        0 \leq \rho \leq 2
    \). Vi har inga villkor på \(
        \theta
    \), alltså är \(
        0 \leq \theta \leq 2\pi
    \). Alltså ges \(
        E
    \) i sfäriska koordinater av att \(
        0 \leq \rho \leq 2
    \), \(
        0 \leq \phi \leq \frac{\pi}{4} 
    \) och \(
        0 \leq \theta \leq 2\pi
    \). 

    Integralen är då 
    \begin{align*}
        \iiint_E {(x^2+y^2+z^2)}^2 \dd V &= \int_0^{2\pi} \int_0^2 \int_0^{\frac{\pi}{4}} {\rho^2}^2 \rho^2 \sin(\phi) \dd \phi \dd \rho \dd \theta \\
            &= \int_0^{2\pi} \int_0^2 \rho^6 \left[ -\cos(\phi) \right]_0^{\frac{\pi}{4}}  \dd \rho \dd \theta \\
            &= \int_0^{2\pi} \int_0^2 \rho^6 (- \cos(\frac{\pi}{4}) - \cos(0)) \dd \rho \dd \theta \\
            &= (- \cos(\frac{\pi}{4}) - \cos(0)) \int_0^{2\pi} \left[ \frac{\rho^7}{7} \right]_0^2 \dd \theta \\
            &= (1 - \frac{1}{\sqrt{2}} ) \int_0^{2\pi} \frac{2^7}{7}  \dd \theta \\
            &= (1 - \frac{1}{\sqrt{2}} )  \frac{2^7}{7} \int_0^{2\pi} 1 \dd \theta \\
            &= (1 - \frac{1}{\sqrt{2}} )  \frac{2^7}{7} \cdot 2\pi \\
            &= (1 - \frac{1}{\sqrt{2}} ) \frac{2^8 \pi}{7} 
    \end{align*}
\end{ex}

\subsection{Vektorfält (16.1)}
Hittils har vi studerat funktioner 
\begin{enumerate}
    \item \(
        \realn^n \to \realn
    \) 

    \item \(
        \realn \to \realn^n
    \) 
\end{enumerate}

Det är naturligt att också studera funktioner \(
    \realn^n \to \realn^m
\).

\begin{defn}[Vektorfält]
    Ett \emph{vektorfält} på \(
        D \subseteq \realn^n
    \) där \(
        n \in \intn
    \) är en funktion \(
        \ve{F} :\; D \to \realn^n
    \), d.v.s.\ att för varje \(
        (x,y) \in D
    \) ger det en vektor \(
        \ve{F} (x,y) \in \realn^n
    \).
\end{defn}

Vektorfält dyker ofta upp i fysikaliska tillämpningar, t.ex.\ 
\begin{enumerate}
    \item Ett \emph{hastighetsfält} \(
        \ve{v} 
    \) är ett vektorfält där \(
        \ve{v} (x,y)
    \) representerar hastigheten i punkten \(
        (x,y)
    \) av något som rör sig med varierande hastighet, t.ex.\ en gas eller 
    en vätska. 

    \item Ett \emph{kraftfält} \(
        \ve{F} 
    \) är ett vektorfält där \(
        \ve{F} (x,y)
    \) representerar kraften på en partikel i punkten \(
        (x,y)
    \), t.ex.\ ett gravitationsfält eller 
    ett elektriskt fält. 
\end{enumerate}

Ett viktigt matematiskt exempel är följande:

\begin{defn}[Gradientfält]
    Om vi har en funktion \(
        f: \; D \to \realn
    \), \(
        D \realn^2 
    \) så är dess \emph{gradientfält} \[
        \nabla f(x,y) = \lang f_x(x,y), f_y(x,y)\rang.
    \] 

    Motsvarande funkar även i \(
        \realn^3
    \).
\end{defn} 

\end{document} 
