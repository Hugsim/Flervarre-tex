\documentclass[a4paper]{article}
\usepackage{../flpack}

\begin{document}

\providecommand\fname{}
\renewcommand\fname{19-09-17}

\subsection{Lokala extremvärden forts.}
\begin{defn}[Kritiska/stationära punkter]
    Om \(
        f_x(a,b) = f_y(a,b) = 0
    \) säger man att \(
        f
    \) har en \emph{kritisk/stationär punkt} i \(
        (a,b)
    \). 
\end{defn}

Alla lokala extremvärden är kritiska punkter (om de partiella derivatorna 
existerar) men det kan finns kritiska punkter som inte är lokala extremvärden.
Sådana punkter kallas \emph{sadelpunkter}.

\begin{ex}
    Låt \(
        f(x,y) = x^2-y^2
    \), bestäm dess kritiska punkter.

    Den har en sadelpunkt i \(
        (0,0)
    \) eftersom \(
        \nabla f = \lang 2x, -2y\rang \implies \nabla f(0,0) = (0,0)
    \), d.v.s.\ \(
        (0,0)
    \) är en kritisk punkt, men den är varken ett lokalt maximum eller minimum.

    Grafen har formen av en \enquote{sadel}.

    \f
\end{ex}

Hur avgör man då om en kritisk punkt är ett extremvärde? I en variabel 
gäller, för funktioner där \(
    f'(a) = 0
\), att om \(
    f''(a) > 0
\) så har \(
    f(x)
\) ett lokalt minimum i \(
    x=a
\) och om \(
    f''(a) < 0
\) så har \(
    f(x)
\) ett lokalt minimum i \(
    x=a
\). Om \(
    f''(a) = 0
\) kan vi inte säga något om punkten.

\begin{sats}[Test för lokala extrempunkter i två variabler]
    Anta att de partiella derivatorna av ordning \(
        2
    \) till \(
        f(x,y)
    \) är kontinuerliga nära \(
        (a,b)
    \) och att \(
        f_x(a,b) = f_y(a,b) = 0
    \).

    Låt \(
        D = f_{xx}(ab)f_{yy}(a,b)-{f_{xy}(a,b)}^2
    \). Då har vi fyra fall:

    \begin{enumerate}
        \item Om \(
            D > 0 \text{ och } f_{xx}(a,b) > 0\text{ (eller } f_{yy}(a,b) > 0\text{)}  
        \) är \(
            (a,b)
        \) ett lokalt minimum.

        \item Om \(
            D > 0 \text{ och } f_{xx}(a,b) < 0\text{ (eller } f_{yy}(a,b) < 0\text{)}  
        \) är \(
            (a,b)
        \) ett lokalt maximum.
    
        \item Om \(
            D < 0
        \) är \(
            (a,b)
        \) en sadelpunkt.

        \item Om \(
            D = 0
        \) vet vi inget om punkten.
    \end{enumerate}

    \begin{proof}
        Går ut på att reducera till motsvarande test i en variabel, se slutet 
        av 14.7 i boken.
    \end{proof}
\end{sats}

Eftersom \(
    f_{xy}(a,b) = f_{yx}(a,b)
\) (Clairaut's sats) är \(
    D = \det{H}
\) där \(
    H 
\) är \enquote{Hessianmatrisen} \(
    H = 
    \begin{bmatrix}
        f_{xx} & f_{xy}\\ 
        f_{yx} &f_{yy} 
    \end{bmatrix}
\).

För att komma ihåg vilket villkor som ger vilken slutsats kan man jämföra med
följande modellfall:
\begin{enumerate}
    \item \(
        f(x,y) = x^2+y^2
    \) som har ett lokalt minimum i origo. \(
        D = 2\cdot 2 - 0 \cdot 0 = 4 > 0
    \), \(
        f_{xx} = 2 > 0
    \).

    \item \(
        f(x,y) = -x^2-y^2
    \) som har ett lokalt maximum i origo. \(
        D = (-2)\cdot(-2) - 0 \cdot 0 = 4 > 0
    \), \(
        f_{xx} = -2 < 0
    \).

    \item \(
        f(x,y) = x^2-y^2
    \) som har en sadelpunkt i origo. \(
        D = 2\cdot (-2) - 0 \cdot 0 = -4 < 0
    \).
\end{enumerate}

\begin{ex}
    Bestäm de kritiska punkterna till \(
        f(x,y) = 3x^2+6xy-y^3
    \) och avgör om de är sadelpunkter, lokala minimum eller lokala maximum.

    Vi börjar med att avgöra kritiska punkter, d.v.s.\ punkter där de partiella 
    derivatorna är \(
        0
    \).

    \(\left\{\begin{matrix}
        f_x = 6x + 6y = 0 \iff x = -y \\
        f_y = 6x - 3y^2 = 0 \iff 2x-y^2 = 0 
    \end{matrix}\right.\) 
    
    Detta ger, om man sätter in uttrycken i varandra, \(
        2(-y)-y^2 = 0 \iff y(-2-y) = 0 \iff y=0 \text{ el. } y = -2
    \). Om vi sätter in detta i uttrycket för \(
        x
    \) får vi \(
        x = -y = -0 = 0
    \) och \(
        x = -(-2) = 2
    \) vilket ger de kritiska punkterna \(
        (0,0)
    \) och \(
        (2,-2)
    \).

    Vi undersöker nu dessa kritiska punkter.

    \(
        f_{xx} = 6
    \), \(
        f_{yy} = -6y
    \) och \(
        f_{xy} = 6
    \). 
    
    I \(
        (0,0) 
    \) gäller \(
        D(0,0) = f_{xx}(0,0) \cdot f_{yy}(0,0) - {f_{xy}(0,0)}^2 = 6 \cdot 0 - 6^2 = -36 \implies (0,0)
    \) är en sadelpunkt.

    I \(
        (2,-2) 
    \) gäller \(
        D(2,-2) = f_{xx}(2,-2) \cdot f_{yy}(2,-2) - {f_{xy}(2,-2)}^2 = 6 \cdot (-6)\cdot (-2) - 6^2 = 72 - 36 = 36 \implies (2,-2)
    \) är en lokal extrempunkt. Eftersom \(
        f_{xx}(2,-2) = 6 > 0
    \) är \(
        (2,-2)
    \) ett lokalt minimum.
\end{ex}

För att generalisera det här till tre eller fler variabler måste man bygga på 
egenvärden till Hessianmatrisen. 

\subsection{Globala extremvärden (14.7)}
I en variabel, om \(
    f(x)
\) är kontinuerlig på ett slutet och begränsat intervall \(
    [a,b]
\) så har \(
    f
\) ett globalt max- och min-värde någonstans i \(
    [a,b]
\). Ej sant för t.ex.\ öppna intervall eller obegränsade intervall. Se 
exempelvis \(
    f(x) = x
\) på intervallet \(
    x \in (0,\infty)
\), då finns inget specifikt max-värde eftersom vi alltid kan gå närmare
\(
    \infty
\) och det finns inget min-värde eftersom vi alltid kan gå lite närmare \(
    0
\).

Vi ska definiera motsvarigheten till \enquote{sluten och begränsad} i 
två variabler.

\begin{defn}[Randpunkter]
    \(
        (a,b)
    \) är en \emph{randpunkt (boundary point)} till \(
        D \subseteq \realn^2
    \) om varje cirkelskiva kring \(
        (a,b)
    \) innehåller både punkter i \(
        D
    \) och punkter som inte är i \(
        D
    \).
\end{defn}

\begin{ex}
    Randpunkter till \(
        D_1 = \{ (x,y) | x^2+y^2 \leq 1 \}
    \) är \(
        \{(x,y) | x^2+y^2 = 1\}
    \).

    Till \(
        D_2 = \{ (x,y) | x^2+y^2 < 1 \}
    \) är randpunkterna precis som ovan \(
        \{(x,y) | x^2+y^2 = 1\}
    \), trots att de inte är i \(
        D
    \).
\end{ex}

\dquote{Här går kanske inte intuitionen jättebra.}{Mr.\ Väsentligen om sin publik}

\begin{defn}[Slutna mängder]
    En mängd \(
        D \subseteq \realn^2
    \) är \emph{sluten} om den innehåller alla sina randpunkter.
\end{defn}

I exemplet ovan är \(
    D_1
\) sluten och \(
    D_2
\) inte sluten.

\begin{defn}
    En mängd \(
        D \subseteq \realn^2
    \) är \emph{begränsad} om den är innehållen i någon cirkelskiva.
\end{defn}

\begin{ex}
    \(
        \{(x,y) | 0 \leq x \leq 1, 0 \leq y \leq 1\}
    \) är begränsad eftersom vi kan innesluta kvadraten i en stor cirkel,
    exempelvis en cirkel med centrum i origo och radie \(
        31415
    \).

    \(
        \{(x,y) | 0 \leq x \leq 1\}
    \) är inte begränsad eftersom den inte går att innesluta i en cirkel.
\end{ex}

\begin{sats}
    Om \(
        f(x,y)
    \) är kontinuerlig på \(
        D \subseteq \realn^2
    \) och \(
        D
    \) är både sluten och begränsad då har \(
        f
    \) globala max- och min-punkter någonstans i \(
        D
    \).

    \begin{proof}
        Beviset involverar definitionen av reella tal vilket är lite lagom 
        överkurs för den här kursen. Finns säkert på nätet eller något.
    \end{proof}
\end{sats}

\end{document} 