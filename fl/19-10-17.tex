\documentclass[a4paper]{article}
\usepackage{../flpack}

\begin{document}

\providecommand\fname{}
\renewcommand\fname{19-10-17}

\subsection{Divergens av ett vektorfält och flöde genom kurvor}
\begin{defn}[Flödet genom en sluten kurva]
    Anta att \(
        C
    \) är en enkel, sluten, positivt orienterad kurva som begränsar ett område \(
        D \subseteq \realn^2
    \) och att \(
        C
    \) parametriseras av \(
        \ver(t) = \lang x(t), y(t)\rang
    \).

    Då kan man visa att \(
        \ven = \frac{\lang y'(t), -x'(t)\rang}{\abs{\ver'(t)}}
    \) är en enhetsnormalvektor till tangentlinjen som pekar ut från \(
        D
    \). (\(
        \ven
    \) fås från \(
        \ver'
    \) genom att rotera \(
        \SI{-90}{\degree}
    \) och sedan normalisera.)

    Som för ytor definieras flödet av \(
        \vef = \lang P, Q \rang 
    \) genom \(
        C
    \) som \(
        \int_C \vef \sprod \ven \dd s
    \). 
\end{defn}

Vi vill nu härleda en smidigare formel för flödet som en dubbelintegral över \(
    D
\).

\begin{påm}[Kurvintegraler av funktioner]
    Kurvintegralen av en funktion \(
        f
    \) över en kurva \(
        C
    \) är \(
        \int_C f \dd s = \int_C f \abs{\ver'(t)}\dd t
    \).
\end{påm}

\begin{påm}[Kurvintegraler av vektorfält]
    Kurvintegralen av ett vektorfält \(
        \vef  = \lang P, Q \rang
    \) över en kurva \(
        C
    \) är \(
        \int_C \vef \sprod \dd \ver = \int_C P \dd x + Q \dd y
    \) där \(
        \dd x = x'(t) \dd t 
    \) och \(
        \dd y = y'(t) \dd t
    \).
\end{påm}

\begin{påm}[Greens sats]
    \(
        \int_C P \dd x + Q \dd y = \iint_D \pdv{Q}{x}-\pdv{P}{y} \dd A
    \).
\end{påm}

Flödet ovan blir då \(
    \int_C \vef \sprod \ven \dd s = \frac{\int_a^b P \cdot y' - Q \cdot x'}{\abs{\ver'}} \cdot \abs{\ver'}\dd t
        = \int_a^b P y' - Qx' \dd t
        = \int_C -Q \dd x + P \dd y
        = \iint_D \pdv{P}{x}+\pdv{Q}{y}\dd A
\). Funktionen innanför återkommer, så vi ger det ett namn.

\begin{defn}[Divergens av ett vektorfält]
    För \(
        \vef = \lang P,Q\rang 
    \) i \(
        \realn^2
    \) är \emph{divergensen} av \(
        \vef
    \) \(
        \text{div } F = \div{\vef} = \pdv{P}{x}+\pdv{Q}{y}
    \).

    För \(
        \vef = \lang P,Q,R\rang 
    \) i \(
        \realn^3
    \) definierar vi \emph{divergensen} av \(
        \vef
    \) som \(
        \text{div } F = \div{\vef} = \pdv{P}{x}+\pdv{Q}{y} + \pdv{R}{z}
    \).
\end{defn}

Då skrivs beräkningen innan definitionen som följer:

\begin{sats}[Divergenssatsen för kurvor]
    \[
        \int_C \vef \sprod \ven \dd s = \iint_D \div{\vef} \dd A
    \] 
\end{sats}

\subsection{Divergenssatsen för ytor}
\begin{defn}[Slutna ytor]
    \(
        S
    \) är en \emph{sluten yta} om den är randen till något begränsat område \(
        E \subseteq \realn^3
    \).
\end{defn}

Vi antar att \(
    S
\) består av ändligt många glatta ytor, t.ex.\ består en kub av sex kvadrater.

\begin{defn}
    En sluten yta \(
        S
    \) är \emph{positivt orienterad} om dess orientering pekar ut från området \(
        E
    \) den begränsar.
\end{defn}

\begin{sats}[Divergenssatsen]
    Låt \(
        E \subseteq \realn^3
    \) vara ett begränsat område med rand \(
        S
    \). Anta att \(
        S
    \) är positivt orienterad och att den består av ändligt många glatta ytor. 
    Anta också att \(
        \vef
    \) är ett vektorfält på \(
        E
    \) med kontinuerliga partiella derivator på \(
        E
    \). 

    Då är \[
        \iint_S \vef \sprod \dd \ve{S} = \iiint_E \div{\vef} \dd V.
    \] 

    Detta motsvarar fallet med kurvor eftersom \(
        \iint_S \vef \sprod \dd \ve{S} = \iint_S \vef \sprod \ven \dd S
    \) 

    \begin{proof}
        Vi visar det i ett specialfall. Beviset i bokens 16.9 bygger på samma idé
        men är lite mer allmänt.

        Låt \(
            E = [a,b] \crossproduct [c,d] \crossproduct [e,f]
        \) och anta att \(
            \vef = \lang 0,0,R\rang
        \).

        Då följer divergenssatsen från vanlig upprepad integration:

        Låt \(
            D = [a,b] \cprod [c,d]
        \) så att \(
            E = D \cprod [e,f]
        \).

        Då är \(
            \div{\vef} = \pdv{R}{z}
        \) och därmed 
        \begin{align*}
            \iiint_E \div{\vef} \dd V &= \iint_D \int_e^f \pdv{R}{z} \dd z \dd A \\
                &= \iint_D \qty[R]_e^f \dd A \\
                &= \iint_D R(x,y,f) \dd A - \iint_D R(x,y,e) \dd A.
        \end{align*}

        Vi jämför nu detta med ytintegralerna i vänsterledet i divergenssatsen.
        Randen \(
            S
        \) till \(
            E
        \) består av sex stycken rektanglar i och med att \(
            E
        \) är ett rätblock. Normalvektorerna behöver peka åt olika håll för att \(
            E
        \) ska vara positivt orienterad. Toppen av rätblocket \(
            S_1
        \) (där \(
            z = f
        \)) kan parametriseras av \(
            \ver_1(x,y) = \lang x,y,f\rang
        \) där \(
            (x,y) \in D
        \) och dess normalvektor är \(
            \ven_1 = \lang 0,0,1\rang
        \).

        Då blir \(
            \iint_{S_1} \vef \sprod \dd \ve{S} = \iint_D \vef(\ver_1(x,y)) \sprod \ven_1 \dd A
                = \iint_D \lang 0,0,R(x,y,f)\rang \sprod \lang 0,0,1\rang \dd A
                = \iint_D R(x,y,f) \dd A
        \) vilket ger första integralen i divergenssatsen.

        På samma sätt kan man hantera botten av rätblocket \(
            S_2
        \) som ger andra integralen. Minustecknet kommer av att dess normalvektor \(
            \ven_2 = \lang 0,0,-1\rang
        \).

        De andra sidorna av kuben kommer inte att ge bidrag eftersom de bara 
        pekar i \(
            x
        \)- eller \(
            y
        \)-led så att deras \(
            z
        \)-koordinat är \(
            0
        \). Då blir skalärprodukten inuti integralerna \(
            0
        \).

        För att sammanfatta blir \(
            \iint_E \div{\vef} \dd V = \iint_{S_1} \vef \sprod \dd \ve{S} 
                    + \iint_{S_2} \vef \sprod \dd \ve{S} 
                    + \cdots
                    + \iint_{S_6} \vef \sprod \dd \ve{S} 
                = \iint_S \vef \sprod \dd \ve{S}
        \) vilket visar divergenssatsen i detta specialfallet.
    \end{proof}
\end{sats}

\begin{ex}
    Beräkna flödet av \(
        \vef = \lang z^2, y^2, x^2 \rang
    \) ut ur området mellan planen \(
        x=0
    \), \(
        x=1
    \), \(
        y=0
    \), \(
        y=2
    \), \(
        z=0
    \), \(
        z=2-2x
    \).

    Området som begränsas av planen är \(
        E = \{ \xyz \; \| \; 0 \leq x \leq 1, 0 \leq y \leq 2, 0 \leq z \leq 2-2x \}
    \). 

    Då får vi att om \(
        S
    \) är den positivt orienterade randen till \(
        E
    \) så är flödet ut ur \(
        E
    \) samma som flödet genom \(
        S
    \) vilket är 
    \begin{align*}
        \iint_S \vef \sprod \dd \ve{S} &= \iiint_E \div{\vef} \dd V \\
            &= \iiint_E 2y \dd V \\
            &= \int_0^1 \int_0^2 \int_0^{2-2x} 2y \dd z \dd y \dd x \\
            &= \dots \\
            &= 4
    \end{align*}
\end{ex}

\end{document}
