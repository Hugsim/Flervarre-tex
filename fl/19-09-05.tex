\documentclass[a4paper]{article}
\usepackage{../flpack}

\begin{document}

\providecommand\fname{}
\renewcommand\fname{19-09-05}

\subsection{Funktioner av flera variabler (14.1)}
\begin{defn}[Funktioner av två variabler, definitionsmängd och värdemängd]
    En (reellvärd) \emph{funktion av två variabler} är en regel 
    som för varje \(
        (x,y) \in D \subseteq \mathbb{R}^2
    \) ger ett reellt tal \(
        f(x,y)
    \). \(
        D 
    \) kallas för \emph{definitionsmängd} (domain) till \(
        f
    \) och \(
        \left\{f(x,y) | (x,y) \in D\right\}
    \) kallas \emph{värdemängden} (range) till \(
        f
    \). Skriver \(
        f : D \to \RR
    \). 
\end{defn}

\begin{ex}
    Mest grundläggande: koordinatfunktionerna \(
        x
    \) och \(
        y
    \).
\end{ex}

\begin{ex}
    Om \(
        f(x,y), g(x,y)
    \) är funktioner kan vi skapa nya funktioner: 
    \begin{align*}
        &f(x,y)+g(x,y) \\
        &f(x,y)\cdot g(x,y) \\
        &\bif{f(x,y)}{g(x,y)} (\text{Endast definierad där } g(x,y) \neq 0) \\
        &\sin(f(x,y))\dots \\
        &\sqrt{f(x,y)}, \ln(f(x,y)), \dots (\text{Endast definierad där } f(x,y) \geq 0 \text{ respektive }  f(x,y) > 0)
    \end{align*}
\end{ex}

\begin{defn}[Graf]
    \emph{Grafen} till \(
        f: D \to \RR, D \subseteq \RR^2
    \) är alla \(
        (x,y,z) \in \RR^3
    \) s.a.\ \(
        z = f(x,y), (x,y) \in D
    \).
\end{defn}

\begin{ex}
    Rita grafen till \(
        f(x,y) = x+y
    \).

    Provar att rita längs några räta linjer:

    \(
        f(x,0) = x, f(x,1) = x+1, f(x,2) = x+2
    \) 

    \(
        f(0,y) = y, f(1,y) = y+1, f(2,y) = y+2
    \) 
    \f
\end{ex}

\subsection{Annat sätt att visualisera en funktion}
\begin{defn}[Nivåkurva]
    En \emph{nivåkurva} till en funktion \(
        f(x,y)
    \) består av alla punkter \(
        (x,y)
    \) s.a.\ \(
        f(x,y) = k 
    \) för ett givet värde \(
        k \in \RR
    \). 
\end{defn}

Ritar man nivåkurvor för olika värden på \(
    k
\) får man en \emph{höjdkarta} (contour map) som kan användas
för att bättre förstå funktionen.

Beskriv nivåkurvorna och rita höjdkartor för olika funktioner.
\begin{ex}
    \(
        f(x,y) = x+2y
    \) 

    \(
        f(x,y) = x+2y = k \iff y = -\bif{x}{2}  + \bif{k}{2} 
    \) vilket är en rät linje med lutning \(
        -\half
    \)  som skär \(
        y
    \)-axeln i \(
        \frac{k}{2} 
    \).
    \f
\end{ex}
\begin{ex}
    \(
        f(x,y) = x^2+y^2
    \) 

    \(
        f(x,y) = x^2+y^2 = k 
    \) ger att om \(
        k < 0
    \) har vi en tom nivåkurva. Anta dock \(
        k \geq 0
    \), då är \(
        x^2+y^2 = k = \left(\sqrt{k}\right)^2
    \) vilket är en cirkel med radie \(
        \sqrt{k}
    \) och origo i \(
        (0,0)
    \).
    \f
\end{ex}
\begin{ex}
    \(
        f(x,y) = e^{xy}
    \) 

    \(
        f(x,y) = e^{xy} = k
    \) ger att om \(
        k \leq 0
    \) har vi en tom nivåkurva. Anta dock \(
        k > 0
    \), då får vi \(
        e^{xy} = k \iff
        xy = \ln(k) 
    \). Nu har vi två fall, 
    om \(
        x \neq 0 
    \) får vi \(
        y = \frac{\ln(k)}{x}
    \) och om \(
        x = 0 
    \) får vi \(
        0 = \ln(k) \iff k = 1
    \).

    \f
\end{ex}

\subsection{Funktioner av tre variabler}
Kan på liknande sätt studera funktioner av tre variabler \(
    f(x,y,z)
\). Dock befinner sig dess graf \(
    w = f(x,y,z)
\) i ett rum med fyra dimensioner vilket är lite svårt att rita.
Däremot är nivåytor \(
    f(x,y,z) = k
\) fortsatt användbara och kan visualiseras i vissa enkla fall.

\begin{ex}
    Beskriv nivåkurvorna till \(
        f(x,y,z) = x^2+y^2+z^2
    \).

    \(
        f(x,y,z) = x^2+y^2+z^2 = \left(\sqrt{x^2+y^2+z^2}\right)^2 
    \) vilket är avståndet från punkten \(
        (x,y,z)
    \) till origo i kvadrat. Då får vi att nivåytorna för varje \(
        k
    \) är antingen tom om \(
        k < 0
    \) och sfären med radie \(
        \sqrt{k}
    \) och centrum i origo om \(
        k \geq 0
    \).
\end{ex}

\end{document}