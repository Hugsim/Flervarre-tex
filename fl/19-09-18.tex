\documentclass[a4paper]{article}
\usepackage{../flpack}

\begin{document}

\providecommand\fname{}
\renewcommand\fname{19-09-18}

\subsection{Globala extremvärden (forts.)}
Om \(
    f(x,y) 
\) har ett globalt max-/min-värde i \(
    (a,b) \in D
\) och \(
    (a,b)
\) inte är en randpunkt måste den vara ett lokalt min/max och därmed också
en kritisk punkt (om \(
    f
\) är deriverbar i \(
    (a,b)
\)). 

\subsubsection{Metod för att bestämma globala max/min}
Låt \(
    f
\) vara en deriverbar funktion på \(
    D \subseteq \realn^2
\) där \(
    D
\) är sluten och begränsad. 

Globala max och min fås genom att jämföra funktionsvärden i följande 
kandidater:
\begin{enumerate}
    \item Kritiska punkter i \(
        D
    \) 
    \item Randpunkter till \(
        D
    \) 
\end{enumerate}

Om man parametriserar randpunkterna till \(
    D
\) med ett antal kurvor räcker det med följande kandidater i punkt \(
    2
\) ovan: 

\begin{enumerate}
    \item[2.1.] Kritiska punkter när man ser \(
        f
    \) som en funktion av en variabel längs randkurvor
    \item[2.2.] \enquote{Hörn}, d.v.s.\ ändpunkter till randkurvorna
\end{enumerate}

\begin{ex}
    Beräkna de globala max-/min-punkterna till funktionen \(
        f(x,y) = x^2+y^2-y
    \) på området \(
        D = \{(x,y) | x^2+y^2 \leq 1, y \geq 0\}
    \).

    Området är en halvcirkelskiva som ganska uppenbart är sluten och begränsad.
    
    \f 

    Vi börjar med att leta kritiska punkter: \(
        \left\{\begin{matrix}
            f_x = 2x = 0\\ 
            f_y = 2y-1 = 0
        \end{matrix}\right. \iff (x,y) = (0; 0,5)
    \). Nu får vi kolla om den kritiska punkten ligger i \(
        D
    \). \(
        0^2+{(0,5)}^2 = 0,25 \leq 1 \text{ och } 0,5 \geq 0
    \), så den ligger i \(
        D
    \). \(
        f(0; 0,5) = -\inv{4}
    \). 

    Nu ska vi titta på randpunkter, så vi börjar med att parametrisera randen.
    Vi delar upp det i det raka strecket (\(
        C_1
    \)) och i halvcirkelbågen (\(
        C_2
    \)) för enkelhetens 
    skull. 

    \(
        C_1: \ve{r}_1 (t) = \lang t, 0\rang 
    \) där \(
        -1 \leq t \leq 1
    \).

    \(
        C_2: \ve{r}_2 (t) = \lang \cos(t), \sin(t) \rang
    \) där \(
        0 \leq t \leq \pi
    \).

    Då tittar vi på kritiska punkter för randkurvorna, och vi börjar med \(
        C_1
    \). \(
        g_1(t) = f(\ve{r}_1(t)) = f(t, 0) = t^2
    \). Kritiska punkter för \(
        g_1
    \) är då \(
        g_1'(t) = 0 \iff t = 0
    \). Det ger kandidaten \(
        g_1(0) = f(0,0) = 0
    \). Nu kör vi på \(
        C_2
    \). \(
        g_2(t) = f(\ve{r}_2(t)) = f(\cos(t), \sin(t)) = 
        \cos^2(t) + \sin^2(t) - \sin(t) = 1 - \sin(t)
    \). Kritiska punkter är då \(
        g_2'(t) = -\cos(t) = 0 \iff t = \frac{\pi}{2}
    \) (Fler punkter egentligen, men den här är den enda i vårt tillåtna 
    intervall). Kandidaten är då \(
        g_2(\frac{\pi}{2}) = f(0,1) = 0
    \).

    Nästa steg är att titta på \enquote{hörnen} \(
        (-1, 0) 
    \) och \(
        (1,0)
    \). \(
        f(-1, 0) = {(-1)}^2 + 0^2 - 0 = 1
    \) och \(
        f(1,0) = 1^2+0^2-0 = 1
    \). 

    Allt som allt är vårt globala max \(
        (\pm 1, 0)
    \) och vårt globala min \(
        (0; 0,5)
    \).
\end{ex}

\subsection{Optimering under bivillkor och Lagrangemultiplikatorer (14.8)}
Vi vill lösa följande slags problem: 
Beräkna max av \(
    4x^2+4y
\) när \(
    (x,y)
\) ligger på enhetscirkeln \(
    x^2+y^2=1
\). En möjlighet är att parametrisera enhetscirkeln, t.ex.\ med \(
    \ve{r} (t) = \lang \cos t, \sin t \rang
\) när \(
    0 \leq t \leq 2\pi 
\) och hitta max av \(
    g(t) = f(\ve{r} (t)) = 4\cos^2(t)+4\sin(t)
\) som vanligt.

En annan möjlighet är att se det som ett fall av optimering under bivillkor.
Låt \(
    f(x,y), g(x,y)
\) vara definierade på någon mängd \(
    D
\). Vi vill hitta max och min av \(
    f(x,y)
\) på \(
    D
\) under förutsättning att \(
    g(x,y) = k
\) där \(
    k
\) är konstant. Villkoret \(
    g(x,y) = k
\) kallas för \emph{bivillkor (constraint)}.

\subsubsection{Metoden med Lagrangemultiplikatorer} 
För att hitta max och min till \(
    f(x,y)
\) under bivillkoret \(
    g(x,y) = k
\) (under förutsättning att extremvärden finns och \(
    \nabla g(x,y) \neq 0
\) på \(
    g(x,y) = k
\)) gör man följande:

\begin{enumerate}
    \item[a)] Hitta alla lösningar \(
        (x,y)
    \) till 
    \(
        \left\{\begin{matrix}
            \nabla f(x,y) = \lambda \cdot \nabla g(x,y) \\ 
            g(x,y) = k
        \end{matrix}\right.
    \) för något \( 
        \lambda
    \) (\(
        \lambda
    \) kallas för Lagrangemultiplikator).

    \item[b)] Jämför \(
        f(x,y)
    \) för alla lösningar \(
        (x,y)
    \) från punkt a. Det största värdet ger max och det minsta ger min.
\end{enumerate}

\begin{anm}
    Förutsättningen att extremvärden finns är uppfylld om \(
        \{g(x,y) = k\} 
    \) är sluten och begränsad. I kursen kommer vi bara att studera problem där
    extremvärden existerar.
\end{anm}

\begin{ex}
    Hitta max och min till \(
        f(x,y) = 4x^2+4y
    \) under bivillkoret \(
        g(x,y) = x^2+y^2 = 1
    \).

    Enligt metoden ska vi då lösa ekvationssystemet 
    \(
        \left\{\begin{matrix}
            \nabla f = \lambda \cdot \nabla g\\ 
            g = 1
        \end{matrix}\right. 
        \iff 
        \left\{\begin{matrix}
            f_x = \lambda \cdot g_x \\
            f_y = \lambda \cdot g_y \\ 
            x^2+y^2 = 1
        \end{matrix}\right.
        \iff 
        \left\{\begin{matrix}
            8x = \lambda \cdot 2x \\
            4 = \lambda \cdot 2y \\
            x^2+y^2 = 1
        \end{matrix}\right.
    \) Den första ekvationen ger \(
        8x = \lambda \cdot 2x \iff 4x = \lambda x \iff x = 0 \text{ el. } \lambda = 4 
    \). Första fallet (\(
        x=0
    \)) ger genom ekvation 3 \(
        y^2 = 1 \iff y = \pm 1
    \) vilket med ekvation 2 ger \(
        \lambda = \frac{4}{2y} = \pm 2
    \). Här får vi komma ihåg att det ger två lösningar \(
        (x,y,\lambda) = (0,1,2) \text{ och } (0,-1,-2)
    \). Andra fallet ger genom ekvation 2 \(
        4 = 4 \cdot 2 \cdot y \iff y = \half
    \). Ekvation 3 ger då \(
        x^2 = 1 - \inv{4} =\frac{3}{4} \iff x = \pm \frac{\sqrt{3}}{2}
    \). Vi kontrollerar att det ger de två lösningarna \(
        (x,y,\lambda) = (\pm \frac{\sqrt{3}}{2}, \half, 4) 
    \). Med det får vi följande lösningar till det ursprungliga problemet:
    \begin{align*}
        f(0,1) &= 4 \\
        f(0,-1) &= -4 \\
        f(\frac{\sqrt{3}}{2}, \half) &= 4 \cdot \frac{3}{4} + 4 \cdot \half = 5 \\
        f(\frac{-\sqrt{3}}{2}, \half) &= 5 
    \end{align*}
    Alltså har vi ett max på \(
        5
    \) i \(
        (\pm\frac{\sqrt{3}}{2}, \half)
    \) och ett min på -4 i \(
        (0, -1)
    \).
\end{ex}

\end{document} 