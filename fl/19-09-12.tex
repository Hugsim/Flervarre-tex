\documentclass[a4paper]{article}
\usepackage{../flpack}

\begin{document}

\providecommand\fname{}
\renewcommand\fname{19-09-12}

\subsection{Tangentplan}
\[
    z = f(a,b) + f_x(a,b)(x-a)+f_y(a,b)(y-b)
\] 

\begin{ex}[Tangentplan]
    Beräkna tangentplanet till \(
        z = f(x,y) = \sqrt{x^2+y^2}
    \) i punkten \(
        (3,4)
    \). Bestäm också lineariseringen av \(
        f
    \) i \(
        (3,4)
    \) och använd den för att approximera \(
        f(3,01; 3,98)
    \).

    Vi börjar med att räkna ut \(
        f(3,4) = \sqrt{3^2+4^2} = \sqrt{25} = 5
    \). Vi behöver också de partiella derivatorna till \(
        f
    \) och deras värden i \(
        (3,4)
    \): \(
        f_x = \bif{2x}{2\sqrt{x^2+y^2}} 
    \), \(
        f_x(3,4) = \bif{3}{5} 
    \), \(
        f_y = \bif{2y}{2\sqrt{x^2+y^2}} 
    \) och \(
        f_y(3,4) = \bif{4}{5} 
    \). 

    Tangentplanet är då \[
        z = 5 + \bif{3}{5} (x-3) + \bif{4}{5} (y-4).
    \] 
    \(
        L(x,y) = 5 + \bif{3}{5} (x-3) + \bif{4}{5} (y-4).
    \) är lineariseringen av \(
        f
    \). Det är en approximation nära \(
        (3,4)
    \). 

    Vi ska uppskatta \(
        f(3,01; 3,98) \approx L(3,01; 3,98) = 5 + \frac{3}{5} (3,01-3) + \frac{4}{5} (3,98-4) 
        = 5 + \frac{3}{5} \cdot 0,01 + \frac{4}{5} \cdot (-0,02) = 4,99
    \). Det verkliga värdet \(
        f(3,01; 3,98) = 4,990040\dots
    \) vilket är en rätt bra approximation.
\end{ex}

\subsection{Differentierbarhet (14.4)}
\subsubsection{En variabel}
I en variabel gäller att om \(
    f'(a)
\) existerar kan man visa (finns i föreläsningsanteckningarna) att \(
    f(x) = L(x) + \epsilon(x)(x-a)
\) där \(
    L(x) = f(a) + f'(a)(x-a)
\) och \(
    \epsilon(x) \to 0 \text{ när } x \to a
\). Det säger att om \(
    f'(a)
\) existerar är \(
    L
\) en bra approximation av \(
    f
\) nära \(
    a
\) för \(
    f(x) \to 0 
\) snabbare än alla linjära funktioner/förstagradspolynom (Och \(
    L
\) är den enda linjära funktionen som uppfyller detta, alltså är \(
    L
\) den \enquote{bästa} approximationen av \(
    f
\) med en linjär funktion.)

\subsubsection{Fler variabler}
\begin{ex}
    Låt \(
        f(x) = \bif{xy}{x^2+y^2} \text{ om } (x,y) \neq (0,0)\text{, } 0 \text{ annars.}  
        f(0,0) = 0
    \), \(
        f_x(0,0) = 0
    \) och \(
        f_y(0,0) = 0
    \). Dess linearisering kring \(
        (0,0)
    \) är då \(
        L(x,y) = 0
    \). Vi kollar exempelvis längs linjen \(
        f(t,t)
    \). \(
        f(t,t) - L(t,t) = \bif{t^2}{t^2+t^2} - 0 = \half \not\to 0
    \), alltså är \(
        L
    \) inte en bra approximation av \(
        f
    \) nära \(
        (0,0)
    \) trots att de partiella derivatorna existerar i den punkten.
\end{ex}

Många resultat bygger på att \(
    L
\) är en bra approximation av \(
    f
\). Vi inför därför följande definition: 
\begin{defn}[Deriverbarhet/Differentierbarhet]
    \(
        f(x,y)
    \) är \emph{deriverbar} eller \emph{differentierbar} i \(
        (a,b)
    \) om \(
        f_x(a,b)
    \) och \(
        f_y(a,b)
    \) existerar och \(
        f(x,y) = f(a,b) + f_x(a,b)(x-a) + f_y(a,b) (y-b) + 
        \epsilon_1(x,y)(x-a) + \epsilon_2(x,y)(y-b)
    \) där \(
        \epsilon_1, \epsilon_2 \to 0 \text{ när } (x,y) \to (a,b).
    \) Speciellt innebär det att \(
        f(x,y) - L(x,y) \to 0 \text{ när } (x,y) \to (a,b).
    \) 
\end{defn}

\begin{sats}[Kontinuitet och deriverbarhet]
    Om \(
        f_x(x,y)
    \) och \(
        f_y(x,y)
    \) existerar och är kontinuerliga nära \(
        (a,b)
    \) är \(
        f
    \) deriverbar i \(
        (a,b)
    \).

    \begin{proof}
        Beviset använder medelvärdessatsen, se appendix F i kursboken för det.
    \end{proof}
\end{sats}

\subsection{Kedjeregeln (14.5)}
\subsubsection{En variabel}
Om \(
    y = f(x)
\) och \(
    x = g(t)
\) är deriverbara så är \(
    \dv{t} f(g(t)) = f'(g(t)) \cdot g'(t)
\). Annorlunda uttryckt \(
    \dv{y}{t} = \dv{y}{x} \dv{x}{t}
\).

\subsubsection{Två variabler}
Låt \(
    z = f(x,y)
\), \(
    x = g(t)
\) och \(
    y = h(t)
\) vara deriverbara. Då är \[
    \dv{t} f(g(t), h(t)) = f_x(g(t), h(t)) \cdot g'(t) + f_y(g(t), h(t)) \cdot h'(t)
\] eller kortare 
\[
    \dv{z}{t} = \pdv{z}{x}\dv{x}{t} + \pdv{z}{y} \dv{y}{t}.
\] Beviset involverar definitionen av deriverbarhet i flera variabler.

\dquote{Nu har de flesta svarat, och några har svarat rätt i alla fall.}{Mr.\ Väsentligen}
\dquote{Då får vi 1+1 vilket är 2.}{Mr.\ Väsentligen}

\subsubsection{Två variabler (variant)}
Låt \(
    z = f(x,y)
\), \(
    x = g(s, t)
\) och \(
    y = h(s,t)
\) vara deriverbara. Då är kedjeregeln \[
    \pdv{z}{s} = \pdv{z}{x}\pdv{x}{s} + \pdv{z}{y}\pdv{y}{s}
\] eller med avseende på \(
    t
\): \[
    \pdv{z}{t} = \pdv{z}{x}\pdv{x}{t} + \pdv{z}{y}\pdv{y}{t}.
\] 
Beviset följer av den andra varianten.

\subsection{Riktningsderivator och gradienter (14.6)}
De partiella derivatorna är derivator längs linjer där \(
    x
\) eller \(
    y
\) är konstant, d.v.s.\ med riktningsvektor \(
    \uvec{i} = \lang 1, 0 \rang \text{ eller } \uvec{j} = \lang 0,1 \rang
\).

\begin{defn}[Riktningsderivata]
    \emph{Riktningsderivatan (directional derivative)} av \(
        f(x,y)
    \) i riktningen \(
        \vec{u} = \lang a, b \rang
    \) där \(
        \ve{u} 
    \) är en enhetsvektor (längd \(
        \sqrt{a^2+b^2} = 1
    \)) ges av \[
        D_{\ve{u} } f(x,y) = \blim_{h \to 0} \bif{f(x+ah, y+bh)-f(x,y)}{h}.
    \]
\end{defn}

\begin{anm}
    \(
        f_x = D_{\uvec{i}} f
    \) och \(
        f_y = D_{\uvec{j}} f
    \) 
\end{anm}

\begin{defn}[Gradient]
    \emph{Gradienten} av \(
        f(x,y)
    \) är \[
        \nabla f = \lang f_x, f_y\rang
    \] 
\end{defn}

\begin{sats}[Deriverbarhet, riktningsvektor och gradient]
    Om \(
        f
    \) är deriverbar är \[
        D_{\ve{u} } f = \nabla f \cdot \ve{u}.
    \] 

    \begin{proof}
        Låt \(
            g(h) = f(x+ah, y+bh) \text{ och } \ve{u} = \lang a, b \rang
        \). Enligt kedjeregeln är \(
            g'(h) = f_x(x+ah, y+bh) \cdot a + f_y(x+ah, y+bh) \cdot b
        \). \(
            g'(0) = f_x(x, y) \cdot a + f_y(x, y ) \cdot b \implies 
            D_{\ve{u} }f(x,y) = \nabla f(x,y) \cdot \ve{u} 
        \).
    \end{proof}
\end{sats}

\end{document} 