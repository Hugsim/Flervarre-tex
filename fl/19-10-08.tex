\documentclass[a4paper]{article}
\usepackage{../flpack}

\begin{document}

\providecommand\fname{}
\renewcommand\fname{19-10-08}

\subsection{Konservativa vektorfält (16.3)}
\begin{sats}[Partiella derivator till vektorfält och konservativa vektorfält]
    Låt \(
        \ve{F} = \lang P, Q \rang 
    \) vara ett vektorfält på en enkelt sammanhängande mängd \(
        D \subseteq \realn^2
    \) så att \(
        \pdv{P}{y} = \pdv{Q}{x}
    \). Då är \(
        \ve{F} 
    \) konservativt.

    \begin{proof}
        Bevisas med hjälp av Greens sats, vilket vi kommer in på snart.
    \end{proof}
\end{sats}

\begin{anm}
    Satsen gäller inte på mängder \(
        D
    \) som inte är enkelt sammanhängande. Viktigt att dubbelkolla när man räknar!
\end{anm}

\subsubsection{Metod för att hitta en potential om den existerar}
Vi vill lösa ekvationssystemet \[
    \left\{\begin{matrix}
        f_x = P & \text{(i)} \\ 
        f_y = Q & \text{(ii)} 
    \end{matrix}\right.
\] vilket görs genom att först ta en lösning till (i), d.v.s.\ 
hittar en primitiv funktion \(
    f_0
\) till \(
    P
\) med avseende på \(
    x
\). Den allmänna lösningen till (i) är då \(
    f(x,y) = f_0(x,y) + g(y)
\). Den stoppar vi in i (ii) för att bestämma \(
    g(y)
\). Alltså är \(
    Q = f_y \pdv{y}(f_0+g) = \pdv{y}(f_0) + g'(y) 
        \iff g'(y) = Q - \pdv{y}(f_0) 
        \iff g(y)
\) är en primitiv funktion till \(
    Q - \pdv{y}(f_0)
\) med avseende på \(
    y
\). 

\begin{obs}
    I detta sista steget måste vi få att \(
        Q - \pdv{y} f_0 
    \) bara beror på \(
        y
    \), annars är något fel. Exempelvis att \(
        \ve{F} 
    \) inte är konservativt eller att man har räknat fel.
\end{obs}

\subsection{Greens sats (16.4)}
\begin{defn}[Styckvis glatta kurvor]
    En kurva \(
        C
    \) är \emph{styckvis glatt} om den kan delas in i ett antal glatta kurvor \(
        C_1, \dots, C_n
    \) där \(
        C_i 
    \) slutar där \(
        C_{i+1}
    \) börjar. 
\end{defn}

\begin{defn}[Kurvintegraler över styckvis glatta kurvor]
    Om en kurva \(
        C
    \) är styckvis glatt definierar vi \(
        \int_C f \dd s = \int_{C_1} f \dd s + \cdots + \int_{C_n} f \dd s
    \) och \(
        \int_C \ve{F} \sprod \dd \ve{r} = \int_{C_1} \ve{F} \sprod \dd \ve{r} + \cdots + \int_{C_n} \ve{F} \sprod \dd \ve{r} 
    \).
\end{defn}

Vi antar i fortsättningen att våra kurvor är styckvis glatta. 

Kom ihåg att kurvintegralen av ett vektorfält längs en kurva \(
    C
\) som parametriseras av \(
    \ver(t)
\) där \(
    a \leq t \leq b
\) beror på kurvan sedd som en \emph{orienterad kurva} som består av dess punkter 
\(
    \{ \ver(t) \; | \; a \leq t \leq b \}
\) och riktningen man går längs kurvan. 
\begin{defn}[Negationen av kurvor]
    Kurvan \(
        -C
    \) består av samma punkter som \(
        C
    \) men går i motsatt riktning. Parametriseras t.ex.\ som \(
        \ve{s}(t) = \ver(b-t) 
    \) där \(
        0 \leq t \leq b-a
    \).
\end{defn}

Från definitionen av vektorfält följer då att \(
    \int_{-C} \ve{F} \sprod \dd \ver = - \int_C \ve{F} \sprod \dd \ver
\). (Däremot är \(
    \int_{-C} f \dd s = \int_C f \dd s
\).)

\begin{defn}[Områden begränsade av en kurva]
    Låt \(
        C
    \) vara en enkel sluten kurva. Då \emph{begränsar} \(
        C
    \) ett område \(
        D
    \), d.v.s.\ \(
        C
    \) är randen till \(
        D
    \).

    Att visa detta är förvånansvärt komplicerat, det visades inte förrän en bit in
    på 1900-talet.
\end{defn}

\begin{defn}[Kurvors orientation]
    En enkel sluten kurva \(
        C
    \) är \emph{positivt orienterad} om den går moturs kring området den begränsar.
    Man kan också säga att området alltid ligger till vänster när man går längs
    \(
        C
    \):s riktning. Annars är den \emph{negativt orienterad} och då är \(
        -C
    \) positivt orienterad.
\end{defn}

\begin{defn}[Notation]
    Anta att \(
        C
    \) ges av \(
        \ver(t)
    \) där \(
        a \leq t \leq b
    \). Låt också \(
        \ve{F} = \lang P, Q\rang
    \) och \(
        \ver(t) = \lang x(t), y(t) \rang
    \). Då skrivs \(
        \int_C \ve{F} \sprod \dd \ver = \int_a^b \ve{F}(\ver(t)) \sprod \ver'(t) \dd t
            = \int_a^b P(x(t), y(t)) x'(t) + Q(x(t), y(t))y'(t) \dd t
    \) också som \(
        \int_C P \dd x + Q \dd y
    \), d.v.s.\ \(
        \dd x = x'(t)
    \) och \(
        \dd y = y'(t)
    \).
\end{defn}

\begin{sats}[Greens sats]
    Om \(
        C
    \) är en enkel sluten positivt orienterad kurva som begränsar området \(
        D
    \) och \(
        P
    \) och \(
        Q
    \) har kontinuerliga partiella derivator är \[
        \int_C P \dd x +  Q \dd y = \iint_D \pdv{Q}{x} - \pdv{P}{y} \dd A.
    \] 

    \begin{proof}
        Den har liknande form som integralkalkylens fundamentalsats, båda formlerna
        säger att integralen av en derivata på något område kan utryckas m.h.a.\ 
        värden på randen. Beviset för satsen när \(
            D
        \) är ett typ I och ett typ II-område följer satsen relativt enkelt från 
        integralkalkylens fundamentalsats och formeln för upprepad integration
        för typ I och typ II-områden. Bokens 16.4 har mer specifikt bevis. 
    \end{proof}
\end{sats}

Greens sats kan alltså användas för att beräkna kurvintegraler som dubbelintegraler 
och tvärtom. 

\begin{sats}[Greens sats och areor]
    Om \(
        \lang P, Q\rang = \lang -y, 0\rang
    \), \(
        \lang 0, x\rang
    \) eller \(
        \lang -\frac{y}{2} , \frac{x}{2}\rang 
    \) är \(
        \pdv{Q}{x} - \pdv{P}{y} = 1
    \) så att \(
        \text{area}(D) = \iint_D 1 = \int_C -y \dd x = \int_c x \dd y = \half \int -y \dd x + x \dd y 
    \).
\end{sats}

\end{document} 