\documentclass[a4paper]{article}
\usepackage{../flpack}

\begin{document}

\providecommand\fname{}
\renewcommand\fname{19-10-01}

\subsection{Trippelintegraler}
\begin{defn}[Trippelintegraler]
    Låt \(
        E \subseteq \realn^3
    \) vara ett område och \(
        g(x,y,z)
    \) vara definierad på \(
        E
    \).   
    
    Då definierar vi trippelintegralen av \(
        g
    \) över \(
        E
    \): \[
        \iiint_E g(x,y,z) \dd V 
    \] på samma sätt som dubbelintegraler, först när \(
        E = [a,b]\times [c,d] \times [e,f]
    \) är ett rätblock m.h.a.\ Riemannsummor och sedan för allmänna 
    begränsade områden genom att reducera till fallet med rätblock. 
\end{defn}

\subsubsection{Upprepad integration i trippelintegraler}
Vi antar att de funktioner vi använder är kontunierliga s.a.\ integralerna 
existerar. Annars blir det svårt.

Om \(
    E = \{ (x,y,z) \; | \; (x,y) \in D, \;e(x,y) \leq z \leq f(x,y) \}
\) där \(
    D \subseteq \realn^2
\) är \(
    \iiint_E g(x,y,z) \dd V = \iint_D \int_{e(x,y)}^{f(x,y)} g(x,y,z) \dd z \dd A
\). 

Om \(
    D = \{ (x,y) \; | \; a \leq x \leq b, \; c(x) \leq y \leq d(x) \}
\) (är ett typ I-område) är \(
    E = \{ (x,y,z) \; | \; a\leq x \leq b, \; c(x) \leq y \leq d(x) ,\;e(x,y) \leq z \leq f(x,y) \}
\). Då är integralen \[
    \iiint_E g(x,y,z) \dd A = \int_a^b \int_{c(x)}^{d(x)} \int_{e(x,y)}^{f(x,y)} g(x,y,z) \dd z \dd y \dd x.
\] 

Motsvarande formler fås om vi byter roller på variablerna. 

\begin{ex}
    Beräkna \(
        \iiint_E z \dd V
    \) där \(
        E 
    \) är den tetraeder som begränsas av planen \(
        x = 0
    \), \(
        y = 0
    \), \(
        z = 0
    \) och \(
        2x + y + z = 2
    \).

    Området mellan ett antal ytor som bestäms av likheter beskrivs typiskt 
    sett av motsvarande olikheter. (D.v.s.\ byt \(
        =
    \) mot \(
        \geq \text{ el. } \leq
    \).) Men vilket håll ska olikheten vara på? Brukar funka att rita en 
    figur. För att göra det enklare att rita kollar vi var den skär de tre axlarna.
    För \(
        x
    \)-axeln får vi när \(
        y = z = 0
    \). Då är \(
        2x + 0 + 0 = 2 \implies x = 1
    \) och därmed innehåller planet punkten \(
        (1,0,0)
    \). På motsvarande sätt får vi \(
        y = 2
    \) och \(
        z = 2
    \) och punkterna \(
        (0,2,0)
    \) och \(
        (0,0,2)
    \). Då får vi följande utritade tetraeder: 

    \f

    För att beskriva \(
        E
    \) med olikheter får vi \(
        E = \{ (x,y,z) \; | \; x \geq 0,\; y \geq 0,\; z \geq 0,\; 2x+y+z \leq 0 \}
    \). Hållen på olikheterna fås genom att kolla punkter i tetraedern. 
    Ex.\ får vi i punkten \(
        (0,0,0)
    \) som ligger i tetraedern \(
        2\cdot 0 + 0 + 0 = 0 
    \) i origo, och \(
        0 \leq 2
    \).

    Genom att kombinera de två sista olikheterna fås \(
        0 \leq z \leq 2 - 2x - y
    \). För att det ska finnas sådana \(
        (x,y,z)
    \) måste \(
        x \geq 0
    \), \(
        y \geq 0 
    \) och \(
        2-2x-y \geq 0
    \). Då får vi följande område:

    \f

    Då får vi att \(
        E = \{ (x,y,z) \; | \; 0 \leq x \leq 1, \; 0 \leq y \leq 2 - 2x, \; 0 \leq z \leq 2 - 2x - y \}
    \). Nu kan vi till slut integrera! 

    \begin{align*}
        \iiint_E z \dd V &= \int_0^1 \int_0^{2-2x} \int_0^{2-2x-y} z \dd z \dd y \dd x \\
                         &= \int_0^1 \int_0^{2-2x} \left[\frac{z^2}{2} \right]_0^{2-2x-y} \dd y \dd x \\
                         &= \half \int_0^1 \int_0^{2-2x} (2-2x-y)^2 \dd y \dd x \\
                         &= \half \int_0^1 \left[ \frac{(2-2x-y)^3}{-3} \right]_{y = 0}^{2-2x} \dd x \\
                         &= \inv{6} \int_0^1 (2-2x)^3 \dd x \\
                         &= \inv{6} \left[ \frac{(2-2x)^4}{4\cdot(-2)} \right]_0^1 \\
                         &= \inv{6} \cdot \frac{2^4}{4\cdot 2} \\
                         &= \inv{3}.
    \end{align*}
\end{ex}

\dquote{Fyra plus två plus två blir åtta.}{Randomsnubbe om trivialiteter}

\subsection{Cylindriska koordinater och trippelintegraler (15.7)}
Cylindriska koordinater betecknas \(
    (r,\theta,z) \in \realn^3
\) och fås genom att byta ut \(
    (x,y)
\) mot de polära koordinaterna \(
    (r,\theta)
\), d.v.s.\ \(
    \left\{\begin{matrix}
        x = r \cos(\theta)\\ 
        y = r \sin(\theta)\\ 
        z = z
    \end{matrix}\right.
\). Om vi har ett område \(
    E = \{ (x,y,z) \; | \; (x,y) \in D, \; e(x,y) \leq z \leq e(x,y) \}
\) där \(
    D \subseteq \realn^2 
\) är en polär rektangel med \(
    a \leq r \leq b
\) och \(
    \alpha \leq \theta \leq \beta 
\) följer av formlerna för upprepad integration av trippelintegraler och 
formeln för integration i polära koordianter för dubbelintegraler att \[
    \iiint_E g(x,y,z) \dd V = \int_a^b \int_\alpha^\beta \int_{e(r\cos(\theta), r\sin(\theta))}^{f(r\cos(\theta), r\sin(\theta))} g(r\cos(\theta), r\sin(\theta)) r \dd z \dd \theta \dd r.
\] 
Kort säger man ofta att \(
    \dd V = r \dd z \dd \theta \dd r
\).

\begin{ex}
    Beräkna \(
        \iiint_E z \dd V
    \) där \(
        E 
    \) är området mellan konen \(
        x^2+y^2=z^2
    \), \(
        z \geq 0
    \) och planet \(
        z = 1
    \). 

    Vi såg förra veckan att om vi låter \(
        D = \{ (x,y) \; | \; x^2+y^2\leq 1\}
    \) är \(
        E = \{ (x,y,z) \; | \; (x,y) \in D,\; \sqrt{x^2+y^2}\leq z \leq 1 \}
    \). \(
        D 
    \) ges i polära koordinater av \(
        0 \leq r \leq 1
    \) och \(
        0 \leq \theta \leq 2\pi
    \). 

    Då får vi  
    \begin{align*}
        \iiint_E z \dd V &= \int_0^1 \int_0^{2\pi} \int_r^1 z \dd z r \dd \theta \dd r \\
                         &= \int_0^1 \int_0^{2\pi} \left[ \frac{z^2r}{2}  \right]_{z=r}^1 \dd \theta \dd r \\
                         &= \half \int_0^1 \int_0^{2\pi} r - r^3 \dd r \dd \theta \dd r \\
                         &= \pi \int_0^1 r - r^3 \dd r \\
                         &= \pi \left[ \frac{r^2}{2} - \frac{r^4}{4} \right]_0^1 \\
                         &= \pi \left( \half - \inv{4} \right) \\
                         &= \frac{\pi}{4}.
    \end{align*}
\end{ex}

\end{document} 