\documentclass[a4paper]{article}
\usepackage{../flpack}

\begin{document}

\providecommand\fname{}
\renewcommand\fname{19-09-06}

\subsection{Gränsvärden i flera variabler (14.2)}
\begin{defn}[Gränsvärden i flera variabler (Formell)]
    En funktion \(
        f(x,y) 
    \) har \emph{gränsvärde} \(
        L
    \) när \(
        (x,y)
    \) går mot \(
        (a,b)
    \), vilket skrivs \(
        \lim_{(x,y) \to (a,b)} f(x,y) = L
    \), om \(
        \forall \epsilon > 0, \exists \delta > 0 \text{ s.a.\ om } 
        0 < \abs{(x,y) - (a,b)} = \sqrt{{(x-a)}^2 + {(y-b)}^2} < \delta  
    \) så är \(
        \abs{f(x,y) - L } < \epsilon
    \).
\end{defn}

\begin{defn}[Gränsvärden i flera variabler (Informell)]
    \(
        f(x,y) 
    \) är hur nära \(
        L
    \) som helst så länge \(
        (x,y)
    \) är tillräckligt nära \(
        (a,b)
    \). Liknande, men mer oprecist: \(
        f(x,y)
    \) närmar sig \(
        L
    \) när \(
        (x,y)
    \) närmar sig \(
        (a,b)
    \).
\end{defn}

\subsubsection{Räkneregler}
Anta \(
    \lim_{(x,y)\to (a,b)} f(x,y) = L
\) och \(
    \lim_{(x,y)\to (a,b)} g(x,y) = M
\). Då gäller 
\begin{align*}
    \lim_{(x,y)\to (a,b)} f(x,y) + g(x,y) &= L + M \\
    \lim_{(x,y)\to (a,b)} f(x,y)\cdot g(x,y) &= L \cdot M \\
    \lim_{(x,y)\to (a,b)} \bif{f(x,y)}{g(x,y)} &= \bif{L}{M} \text{ (Endast om } g(x,y) \neq 0\text{)} \\
    \lim_{(x,y)\to (a,b)} x &= a \\
    \lim_{(x,y)\to (a,b)} y &= b 
\end{align*}
Om \(
    h(t)
\) en funktion och \(
    \blim_{t \to L} h(t) = K
\) så är \[
    \lim_{(x,y)\to (a,b)} h(f(x,y)) = K.
\]

Instängningsregeln/Squeeze theorem: Om \(
    \forall x, y: f(x,y) \leq g(x,y) \leq h(x,y)
\) och \(
    \lim_{(x,y)\to (a,b)} f(x,y) = L = \lim_{(x,y)\to (a,b)} h(x,y)
\) gäller \[
    \lim_{(x,y)\to (a,b)} g(x,y) = L.
\] 

\begin{ex}
    Beräkna \(
        \lim_{(x,y)\to (0,0)} \bif{x^4-y^4}{x^2+y^2} 
    \) om det existerar. 
    
    Naivt kan man säga \(
        \lim_{(x,y)\to (a,b)} x^2+y^2 = 0^2+0^2 = 0
    \) och därmed anta att gränsvärdet inte existerar.
    Dock kan vi förenkla 
    \begin{align*}
        & \lim_{(x,y)\to (0,0)} \bif{x^4-y^4}{x^2+y^2} \\
       =& \lim_{(x,y)\to (0,0)} \bif{(x^2+y^2)(x^2-y^2)}{x^2+y^2} \\
       =& \lim_{(x,y)\to (0,0)} x^2-y^2 \\
       =& 0^2 - 0^2 \\
       =& 0.
    \end{align*}
\end{ex}

\begin{ex}
    Visa att \(
        \lim_{(x,y)\to (0,0)} \bif{x^2y^2}{x^2+y^2} = 0 
    \). 

    Inse först att \(
        \forall x, y: 0 \leq \bif{x^2y^2}{x^2+y^2} \text{ och } \bif{x^2y^2}{x^2+y^2} = y^2 \bif{x^2}{x^2+y^2} \leq y^2 \bif{x^2+y^2}{x^2+y^2} = y^2
    \). Enligt instängingsregeln är \(
        \lim_{(x,y)\to (0,0)} \bif{x^2y^2}{x^2+y^2} = 0
    \) eftersom \(
        0 \leq \bif{x^2y^2}{x^2+y^2} \leq y^2
    \) och \(
        \lim_{(x,y)\to (0,0)} 0 = \lim_{(x,y)\to (0,0)} y^2 = 0
    \).
\end{ex}

\begin{ex}
    Låt \(
        g(x,y) = \bif{\ln(x^2+y^2)}{x^2+y^2-1} 
    \). Beräkna \(
        \lim_{(x,y) \to (0,1)} g(x,y)
    \) om det existerar.

    Om \(
        f(x,y) = x^2+y^2
    \) och \(
        h(t) = \bif{\ln(t)}{t-1} 
    \) så är \(
        g(x,y) = h(f(x,y))
    \). Eftersom \(
        \lim_{(x,y) \to (0,1)} f(x,y) = 0^2 + 1^2 = 1
    \) får vi \(
        \lim_{(x,y) \to (0,1)} g(x,y) = \lim_{t \to 1} h(t) 
        = \lim_{t \to 1} \bif{\ln(t)}{t-1} = 1
    \). Sista steget är ett standardgränsvärde eller går att räkna 
    ut med l'Hospitals regel. 
\end{ex}

Om \(
    \lim_{(x,y) \to (a,b)} f(x,y) = L
\) måste gränsvärdet av \(
    f
\) när man närmar sig punkten \(
    (a,b)
\) längs alla räta linjer genom \(
    (a,b)
\) vara \(
    L
\). Sådana räta linjer ges av \(
    \ve{r} (t) = \lang a+kt, b+lt \rang 
\). \(
    \ve{r} (t) \text{ går mot } (a,b) \text{ när } t \to 0
\). Detta ger en metod för att se när gränsvärden inte existerar
och vad gränsvärdet måste vara om det existerar,
men inte att det existerar.

\begin{ex}
    Beräkna \(
        \lim_{(x,y) \to (0,0)} \bif{x^2-y^2}{x^2+y^2} 
    \) om det existerar.

    Låt \(
        f(x,y) = \bif{x^2-y^2}{x^2+y^2}
    \). Vi prövar vad som händer när \(
        (x,y) \to (0,0)
    \) längs \(
        x
    \)-axeln och \(
        y
    \)-axeln.

    \(
        x
    \)-axeln: \(
        \ve{r}_1 (t) = \lang t, 0\rang \to \lang 0, 0\rang \text{ när } t \to 0
    \). \(
        f(\ve{r}_1(t) ) = f(t,0) = \bif{t^2-0^2}{t^2+0^2} = 1 \text{ när } t \to 0
    \).

    \(
        y
    \)-axeln: \(
        \ve{r}_2 (t) = \lang  0, t \rang \to \lang 0, 0\rang \text{ när } t \to 0
    \). \(
        f(\ve{r}_2(t) ) = f(0,t) = \bif{0^2-t^2}{0^2+t^2} = -1 \text{ när } t \to 0
    \). 

    Eftersom dessa gränsvärden är olika existerar inte gränsvärdet 
    för \(
        f(x,y)
    \).
\end{ex}

\subsection{Kontinuitet}
\begin{defn}[Kontinuitet i en punkt]
    En funktion \(
        f(x,y)
    \) är \emph{kontinuerlig} i en punkt \(
        (a,b)
    \) om \(
        \lim_{(x,y) \to (a,b)} f(x,y) = f(a,b)
    \).
\end{defn}

\begin{defn}[Kontinuitet på ett område]
    \(
        f
    \) är kontinuerlig på ett område \(
        D \subseteq \RR^2
    \) om \(
        f
    \) är kontinuerlig på alla punkter \(
        (a,b) \in D
    \). Man säger att \(
        f
    \) är kontinuerlig om den är kontinuerlig för alla punkter i
    sin definitionsmängd.
\end{defn}

\end{document}