\documentclass[a4paper]{article}
\usepackage{../flpack}

\begin{document}

\providecommand\fname{}
\renewcommand\fname{19-09-30}

\subsection{Densitet, massa och masscentrum (15.4)}
Låt \(
    D 
\) vara en tunn skiva (lamina) som representeras av \(
    D \subseteq \realn^2
\) med densitet \(
    \rho(x,y)
\). \(
    \rho(x,y)
\) är gränsvärdet när \(
    \epsilon \to 0
\) för densiteten av en kvadrat \(
    A_\epsilon
\) med sidlängd \(
    \epsilon
\) och centrum i \(
    (x,y)
\). 

Om \(
    E
\) är ett litet område kring punkten \(
    (x,y)
\) bör \(
    \rho(x,y) 
\) vara detsamma oberoende av \(
    E
\). För ett sådant område blir då 
\begin{equation}\label{eq:mass}
    \text{massa}(E) \approx \rho(x,y)\cdot \text{area}(E).
\end{equation}

M.h.a.\ \(
    \rho
\) vill vi beskriva massan på och masscentrum i \(
    D
\) ovan.

\subsubsection{Massa}
Vi börjar med att anta att \(
    D
\) är en rektangel med \(
    \rho(x,y) = 0
\) om \(
    (x,y) \not\in D
\). Då delar vi in \(
    D
\) i \(
    M\times N
\) stycken lika stora delrektanglar \(
    R_{ij}
\) där \(
    i \in \{1, \dots, M\}
\) och \(
    j \in \{1, \dots, N\}
\) med area \(
    \Delta R
\) och \(
    (x_{ij}, y_{ij}) \in R_{ij}
\). Om \(
    M
\) och \(
    N
\) är stora blir varje \(
    R_{ij}
\) litet, så att från ekvation~\ref{eq:mass}
ovan får vi \(
    \text{massa}(R_{ij}) \approx \rho(x_{ij}, y_{ij}) \cdot \Delta R 
\). Total massan är då \(
    m = \sum_{m=1}^M \sum_{n=1}^N \text{massa}(R_{ij})
        \approx  \sum_{m=1}^M \sum_{n=1}^N \rho(x_{ij}, y_{ij}) \cdot \Delta R 
        = \iint_D \rho(x, y) \dd A
\). Uppskattningen för massan blir också bättre och bättre när \(
    M, N \to \infty
\). 

\begin{sats}[Massa för område]
    För en skiva \(
        D \subseteq \realn^2
    \) med densitet \(
        rho(x,y)
    \) i varje punkt \(
        (x,y)
    \) är den totala massan \(
        m = \iint_D \rho(x,y) \dd A
    \).

    \begin{proof}
        Bevis finns ovan.
    \end{proof}
\end{sats}

\subsubsection{Masscentrum}
\begin{defn}[Moment]
    \emph{Momentet} med avseende på \(
        x
    \)-axeln kring en punkt \(
        x = a
    \) för en partikel med position \(
        (x,y)
    \) och massa \(
        m
    \) är \(
        m (x-a)
    \). 
\end{defn}

\begin{defn}[Jämvikt av moment]
    En samling partiklar med positioner \(
        (x_i, y_i)
    \) och massor \(
        m_i
    \) där \(
        i \in \{1, \dots, n\}
    \) är i \emph{jämvikt} kring \(
        x = a
    \) om dess totala moment \(
        \sum_{i=1}^n m_i (x_i - a) = 0
    \).

    Motsvarande definition funkar även i \(
        y
    \)-led.
\end{defn}

\begin{sats}[Totalt moment för en skiva och jämvikt]
    Låt \(
        D \subseteq \realn^2
    \) vara en tunn skiva med densitet \(
        \rho(x,y)
    \).

    Dess totala moment med avseende på \(
        x
    \)-axeln kring \(
        x = a
    \) är då \(
        \iint_D (x-a)\rho(x,y) \dd A
    \).

    Samma sak funkar igen även för \(
        y
    \)-led.

    Om \(
        \iint_D (x-a)\rho(x,y) \dd A 
            = \iint_D x \rho(x,y) \dd A - a \iint_D \rho(x,y) \dd A 
            = 0
    \) är skivan i jämvikt.

    \begin{proof}
        Härleds på samma sätt som för massan.
    \end{proof}
\end{sats}

\begin{defn}[Moment kring \(x\)-axeln]
    Momentet med avseende på \(
        x
    \)-axeln kring \(
        x=0
    \) är \(
        M_x = \iint_D x \rho(x,y) \dd A
    \).
\end{defn}

\begin{sats}[Masscentrum]
    För en skiva \(
        D \subseteq \realn^2
    \) med densitet \(
        \rho(x,y)
    \) är dess \emph{masscentrum} \(
        (\overline{x}, \overline{y})
    \) där \(
        \overline{x} = \frac{M_x}{m} = \inv{m} \iint_D x \rho(x,y) \dd A
    \) och \(
        \overline{y} = \frac{M_y}{m} = \inv{m} \iint_D y \rho(x,y) \dd A
    \).
\end{sats}

\begin{ex}
    Beräkna massan och masscentrum för triangeln \(
        D
    \) med hörn i \(
        (0,0)
    \), \(
        (1,0)
    \) och \(
        (0,1)
    \) med konstant densitet \(
        \rho(x,y) = 1
    \).

    \(
        m = \iint_D \rho(x,y) \dd A
            = \iint 1 \dd A
            = \half
    \).

    För att räkna ut \(
        M_x
    \) måste vi kunna beskriva randerna till \(
        D
    \). Vi noterar att hypotenusan till triangeln beskrivs av \(
        y = 1-x
    \) så att \(
        D \{ (x,y) \; | \; 0 \leq x \leq 1, 0 \leq y \leq 1-x\}
    \) och därmed 
    \begin{align*}
        M_x &= \iint_D x \rho(x,y) \dd A\\
            &= \iint_d x \dd A\\
            &= \int_0^1 \int_0^{1-x} x \dd y \dd x\\
            &= \int_0^1 x \int_0^{1-x} 1 \dd y \dd x \\
            &= \int_0^1 x\cdot(1-x) \dd x \\
            &= \int_0^1 x-x^2 \dd x  \\
            &= \left[ \frac{x^2}{2} - \frac{x^3}{3} \right]_0^1  \\
            &= \half - \inv{3}  \\
            &= \inv{6}.
    \end{align*}

    Då har vi \(
        x
    \)-koordinaten för masscentrum \(
        \overline{x} = \frac{M_x}{m} = \frac{M_x}{\half} = \frac{\inv{6}}{\inv{2}} = \inv{3}
    \).

    I och med att triangeln är symmetrisk måste \(
        M_y = M_x = \inv{3}
    \) och därmed är masscentrum \(
        (\overline{x}, \overline{y}) = (\inv{3}, \inv{3})
    \). Man kan självklart räkna ut \(
        M_y
    \) på samma sätt, men det behövs inte här.
\end{ex}

\begin{sats}[Masscentrum för triangel]
    I en triangel, om densiteten är konstant, är masscentrumet skärningspunkten
    för triangelns medianer, d.v.s.\ linjerna från triangelns hörn till mitten
    av motstående sida.
\end{sats}

\subsection{Medelvärden}
\begin{sats}[Medelvärde av en funktion]
    Låt \(
        f(x,y)
    \) vara definierad på en mängd \(
        D \subseteq \realn^2
    \). Då är dess \emph{medelvärde} \(
        \frac{\iint_D f(x,y) \dd A}{\iint_D 1 \dd A}
        = \frac{\iint_D f(x,y) \dd A}{\text{area}(D)} 
    \).
\end{sats}

Speciellt, om \(
    D
\) är en skiva med konstant densitet \(
    \rho(x,y) = 1
\) är komponenterna i dess masscentrum \(
    (\overline{x}, \overline{y})
\) medelvärdet av \(
    x
\) och \(
    y
\) på \(
    D
\).

\subsection{Arean av grafen till en funktion (15.5)}
Låt \(
    f(x,y)
\) vara definierad på \(
    D
\), då ger den en yta (dess graf) \(
    z=f(x,y)
\) där \(
    (x,y) \in D
\). 

\begin{sats}[Arean av en graf]
    \(
        A = \iint_D \sqrt{1+f_x^2 + f_y^2} \dd A
    \).

    \begin{proof}
        Inte idag. % Beviset lämnas som övning till läsaren.
    \end{proof}
\end{sats}

Jämför med längden av en graf \(
    y=f(x)
\) där \(
    a \leq x \leq b
\): \(
    L = \int_a^b \sqrt{1+{f'(x)}^2} \dd x
\).

Nästa vecka pratar vi om en formel för mer allmänna ytor och kort om varför
den ser ut som den gör.

\begin{ex}
    Bestäm arean av grafen \(
        z = f(x,y)
    \) där \(
        (x,y) \in D
    \), \(
        f(x,y) = \frac{2}{3}( x^{\frac{3}{2}} + y^{\frac{3}{2}} )
    \) och \(
        D = [0,1] \times [0,2]
    \).

    \(
        f_x = \frac{2}{3} \frac{3}{2} x^{\half} = \sqrt{x}
    \). På samma sätt blir \(
        f_y = \sqrt{y}
    \).

    \(
        \sqrt{ 1 + f_x^2 + f_y^2 } = \sqrt{1 + \sqrt{x}^2 + \sqrt{y}^2} 
            = \sqrt{1+x+y} 
    \) ger \(
        A \iint_D \sqrt{1+x+y} \dd A
            = \int_0^1 \int_0^2 \sqrt{1+x+y} \dd y \dd x
            = \cdots = \frac{4}{15} (33 - 4 \sqrt{2} - 9 \sqrt{3})
    \).
\end{ex}

\end{document} 