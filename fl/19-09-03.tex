\documentclass[a4paper]{article}
\usepackage{../flpack}

\begin{document}

\providecommand\fname{}
\renewcommand\fname{19-09-03}

\subsection{Inledning}
Punkter i planet \(
    \mathbb{R}^2
\) beskrivs med två koordinater, oftast \(
    (x,y)
\). Punkter i rummet \(
    \mathbb{R}^2
\) beskrivs med tre koordinater, oftast \(
    (x,y,z)
\). 

I envariabelanalys studerar man funktioner \(
    f: \mathbb{R} \to  \mathbb{R}
\), då är det naturligt att också studera vektorvärda funktioner \(
    f: \mathbb{R}^m \to  \mathbb{R}^n
\). \(
    f = \lang f_1(x_1, \dots, x_m), \dots, f_n(x_1, \dots, x_m)\rang 
\) har \(
    n
\) komponenter och \(
    m
\) variabler. Detta är dock för allmänt, 
i kursen ska vi studera viktiga specialfall: 

\(
    n = 2,3, m = 1: f:\mathbb{R} \to  \mathbb{R}^2 
\) så att \(
    f(t)=\lang f_1(t), f_2(t)\rang 
\). En variabel \(
    t
\) bestämmer två komponenter \(
    f_1(t)
\) och \(
    f_2(t)
\).

\(
    n = 1, n = 2,3: f: \mathbb{R}^2 \to  \mathbb{R}, f(x,y) \in \mathbb{R}
\) så att två variabler \(
    x,y
\) bestämmer en komponent/tal \(
    f(x,y)
\). 

\(
    n = m = 2,3: f:\mathbb{R}^2 \to  \mathbb{R}^2
\). Dessa kallas vektorfält.

\subsection{Kurvor i rummet eller planet (13.1)}
\begin{defn}[Kurvor i rummet och deras parametrisering]
En \emph{kurva} \(
    C
\) i planet är alla punkter vars position ges av \(
    \ve{r}(t) = \lang x(t),y(t)\rang 
\) där \(
    x(t), y(t)
\) är kontinuerliga funktioner på något intervall.
\(
    \ve{r}(t) 
\) kallas en \emph{parametrisering} av \(
    C
\). 
\end{defn}

Motsvarande definition av kurvor i rummet.

\begin{ex}
En rät linje genom en punkt \(
    (x_0,y_0)
\) med riktningsvektor \(
    \ve{v} = \lang v_1,v_2\rang  
\) är en kurva som kan parametriseras av \(
    \ve{r}(t) = \lang x_0+tv_1, y_0+tv_2\rang
\).
\end{ex}

Om \(
    \ve{r}(t) = \lang x(t), y(t)\rang 
\) är definierad för \(
    t \text{ nära } t_0
\) ges \emph{gränsvärdet} av \(
    \ve{r}(t) 
\) när \(
    t
\) går mot \(
    t_0
\) av \(
    \blim_{t \to  t_0} \ve{r} (t) = \lang \blim_{t \to  t_0} x(t), \blim_{t \to  t_0} y(t)\rang
\) om gränsvärdena i högerledet existerar.

\subsection{Derivator av vektorvärda funktioner och tangentlinjer (13.2)}
\begin{defn}[Derivatan av vektorvärda funktioner]
    \emph{Derivatan} av \(
        \ve{r} (t) = \ve{r}'(t) = \blim_{h\to  0} \bif{\ve{r} (t+h) - \ve{r}(t) }{h}   
    \) om gränsvärdet existerar.
\end{defn}

Om \(
    \ve{r} (t) = \lang x(t), y(t)\rang
\) gäller att om \(
    x(t) \text{ och } y(t)
\) är deriverbara så är \(
    \ve{r}' (t) = \lang \blim_{h\to  0} \bif{x(t+h) - x(t) }{h}, \blim_{h\to  0} \bif{y(t+h) - y(t) }{h}\rang
    = \lang x'(t), y'(t)\rang
\) 

\begin{ex}
    Låt \(
        \ve{r} (t) = \lang 2t, \sqrt{t}\rang
    \). Beräkna \(
        \ve{r} '(t)
    \).
\end{ex}
\(
    \ve{r} (t) = \lang \dv{t} (2t), \dv{t} (\sqrt{t}) \rang 
    = \lang 2, \half t^{\half-1} \rang 
    = \lang 2, \bif{1}{2 \sqrt{t}} \rang
\) 

\begin{defn}[Tangentvektor och tangentlinje till linjer]
    \(
        \ve{r}'(t)
    \) kallas för \emph{tangentvektor} till \(
        \ve{r}(t)
    \) och linjen genom \(
        P = \ve{r} (t_0)
    \) med riktningsvektor \(
        \ve{r} '(t) 
    \) kallas för \emph{tangentlinjen} till \(
        \ve{r} 
    \) i \(
        P
    \).
    Tangentlinjen ges av \(
    L(t) = \ve{r} (t_0) + t \ve{r} '(t_0)
\).
\end{defn}

\begin{ex}
    Bestäm tangentlinjen till \(
        \ve{r} (t) = \lang 2t, \sqrt{t} \rang
    \) i punkten \(
        P = \ve{r} (4) = \lang 2 \cdot 4, \sqrt{4} \rang
        = \lang 8, 2 \rang
    \). Tangentvektor \(
        \ve{r} '(t) = \lang 2, \inv{2 \sqrt t} \rang
        \implies \ve{r} '(4) = \lang 2, \inv{4} \rang
    \). Tangentlinjen \(
        L(t) = \lang 8, 2 \rang + t \cdot \lang 2, \inv{4} \rang 
        = \lang 8 + 2t, 2 + \bif{t}{4} \rang
    \) 
\end{ex}

\end{document}