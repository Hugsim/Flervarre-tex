\documentclass[a4paper]{article}
\usepackage{../flpack}

\begin{document}

\providecommand\fname{}
\renewcommand\fname{19-10-15}

\subsection{Ytintegraler av funktioner (16.7)}
På ett liknande sätt som för kurvintegraler av funktioner definierar vi 
ytintegraler av funktioner som gränsvärden av Riemannsummor \(
    \sum_{i=1}^m \sum_{j=1}^n f(\ver(u_{ij}^*, v_{ij}^*)) \text{area}(S_{ij})
\) och med liknande härledning som för arean fås följande formel för ytintegralen
av en funktion:

\begin{sats}[Ytintegraler av funktioner]
    Låt \(
        f
    \) vara definierad på någon yta \(
        s
    \) som parametriseras av \(
        \ver(u,v)
    \), \(
        (u,v) \in D
    \). Då är ytintegralen av \(
        f
    \) över \(
        S
    \) \[
        \iint_S f(x,y,z) \dd S = \iint_D f(\ver(u,v)) 
            \abs{\ver_u(u,v) \times \ver_v(u,v)} \dd A.
    \] 
    Kort skriver vi att \(
        \dd S = \abs{\ver_u \times \ver_v} \dd A
    \).
\end{sats}

\begin{sats}[Massa och masscentrum för en yta]
    Om en yta \(
        S
    \) har densitet \(
        \rho\xyz 
    \) är dess massa \(
        m = \iint_S \rho\xyz \dd S
    \) och dess masscentrum \(
        (\overline{x}, \overline{y}, \overline{z}) = (\frac{M_x}{x}, \frac{M_y}{y}, \frac{M_z}{z} )
    \) där \(
        M_x = \iint_S x \rho\xyz \dd S
    \).
\end{sats}

\subsection{Flödesintegraler (16.7)}
Vi vill definiera ytintegralen för ett vektorfält \(
    \vef 
\) över en yta \(
    S
\) så att det ger flöde, d.v.s.\ om \(
    \vef
\) är ett hastighetsfält för exempelvis en vätska är flödet volymen av all vätska
som passerar \(
    S
\) per tidsenhet. 

\subsection{Orientering av ytor}
Möbiusbandet är en speciell yta som fås genom att ta en remsa, vrida ett halvt 
varv och sätta ihop. Den har bara en sida vilket är coolt. 
Flödet beror på riktningen som man passerar ytan, alltså måste man ha en fram-
och en baksida eller en ut- och en insida, men det finns ju då inte på ett 
Möbiusband. 

\begin{påm}
    Om \(
        S
    \) är en yta som parametriseras av \(
        \ver(u,v)
    \) är \(
        \ve{n} = \frac{\ver_u(u,v) \times \ver_v(u,v)}{\abs{\ver_u(u,v) \times \ver_v(u,v)}} 
    \) en enhetsnormalvektor till tangentplanet till \(
        S
    \) i \(
        \ver(u,v)
    \).
\end{påm}

Varje tangentplan har två enhetsnormalvektorer \(
    \pm \ve{n}
\) som pekar åt motsatta håll. 
\begin{defn}[Orienterade ytor]
    En yta sägs vara \emph{orienterad} om man för varje \(
        \xyz
    \) på \(
        S
    \) kan välja en enhetsnormalvektor \(
        \ve{n}\xyz
    \) så att \(
        \ve{n}
    \) varierar kontinuerligt på \(
        S
    \). \(
        \ve{n}
    \) kallas för en \emph{orientering} av \(
        S
    \).
\end{defn}

\begin{ex}
    Ett Möbiusband är inte en orienterad yta.
\end{ex}

\begin{sats}[Orientering för en yta]
    Om \(
        S
    \) är en glatt orienterad yta som parametriseras av \(
        \ver(u,v)
    \) är \(
        \ve{n} = \frac{\ver_u \times \ver_v}{\abs{\ver_u\times\ver_v}} 
    \) en orientering av \(
        S
    \).
\end{sats}

Om \(
    S
\) är grafen \(
    z = f(x,y)
\) kan den orienteras genom att ta orienteringen som pekar uppåt i positiv \(
    z
\)-riktning. Om \(
    S
\) parametriseras av \(
    \ver(x,y) = \lang x,y,f(x,y)\rang
\) är den orienteringen \(
    \ve{n} = \frac{\ver_x\times\ver_y}{\abs{\ver_x\times\ver_y}} 
        = \frac{\lang -f_x, -f_y, 1\rang}{\sqrt{1 + f_x^2 +f_y^2}}
\).

\subsection{Flöde genom en yta}
\begin{sats}[Flöde genom en yta]
    Om \(
        S
    \) är en orienterad yta med orientering \(
        \ve{n}\xyz
    \) och \(
        \ve{v}\xyz
    \) är ett hastighetsfält härleder vi via lämpliga Riemannsummor där \(
        S
    \) approximeras med rektanglar (finns i 16.7 i boken) att \emph{flödet} genom \(
        S
    \) är \[
        \iint_S \ve{v} \sprod \ve{n} \dd S.
    \]
\end{sats}

\begin{sats}[Ytintegralen av ett vektorfält över en yta]
    Låt \(
        \vef
    \) vara ett vektorfält på ytan \(
        S
    \) som har orientering \(
        \ve{n}
    \). Då är \emph{ytintegralen} \[
        \iint_S \vef \sprod \dd \ve{S} = \iint_s \vef \sprod \ve{n} \dd S
    \] 
\end{sats}

Ytintegralen kallas också för \emph{flödesintegral} (flux integral) och dess 
värde kallas \emph{flödet av \(
    \vef
\) genom \(
    S
\)} eftersom det är vad ytintegralen ger när \(
    \vef
\) är ett hastighetsfält.

\begin{sats}[Direkt formel för flödesintegraler]
    Låt \(
        S
    \) vara en glatt orienterad yta som parametriseras av \(
        \ver(u,v)
    \), \(
        (u,v) \in D
    \). 
    \begin{påm}
        \(
            \iint_S f \dd S = \iint_D f \abs{\ver_u \times \ver_v} \dd A
        \) 
    \end{påm}

    \begin{påm}
        \(
            \ven = \frac{\ver_u \times \ver_v}{\abs{\ver_u \times \ver_v}}
        \) ger en orientering av \(
            S
        \).
    \end{påm}

    Med orienteringen ovan blir \(
        \iint_S \vef \sprod \dd \ve{S} = \iint_S f \sprod \ven \dd S 
            = \iint_D \vef \sprod \frac{\ver_u \times \ver_v}{\abs{\ver_u \times \ver_v}} 
                \abs{\ver_u \times \ver_v} \dd A 
            = \iint_D \vef \sprod (\ver_u \times \ver_v) \dd A
    \). 

    Formeln gäller också med motsatt orientering om man sätter ett minustecken 
    framför. 

    Kort säger vi att \(
        \dd \ve{S} = \ver_u \times \ver_v \dd A
    \).
\end{sats}

\begin{ex}
    Beräkna flödet av vektorfältet \(
        \vef = \lang zx, zy, 0\rang
    \) genom halvcylindern \(
        \ver(u,v) = \lang \cos u, \sin u, v \rang
    \) där \(
        (u,v) \in D = \qty[ -\frac{\pi}{2} , \frac{\pi}{2}  ] \times \qty[ 0,1 ]
    \) och halvcylindern är orienterad i positiv \(
        x
    \)-riktning. 

    För att beräkna det måste vi först räkna ut \(
        \ver_u \times \ver_v = 
        \begin{vmatrix}
            \uvec{i} & \uvec{j} & \uvec{k} \\ 
            -\sin u & \cos u & 0 \\ 
            0 & 0 & 1
        \end{vmatrix}
        = \lang \cos u, \sin u, 0 \rang 
    \).

    Om \(
        \frac{-\pi}{2} \leq u \leq \frac{\pi}{2} 
    \) är \(
        \cos u \geq 0
    \), alltså pekar den här i positiv \(
        x
    \)-riktning, så den har orienteringen vi ville ha.

    \(
        \vef(\ver(u,v)) = \vef(\cos u, \sin u, v) 
            = \lang v \cos u, v \sin u, 0 \rang
    \).

    \(
        \vef(\ver(u,v)) \sprod (\ver_u \times \ver_v) 
            = \lang v \cos u, v \sin u, 0\rang \sprod \lang \cos u, \sin u, v \rang
            = v ( \cos^2 u + \sin^2 u)
            = v
    \).

    Alltså är flödet \(
        \iint_S \vef \sprod \dd \ve{S} 
            = \iint_D \vef(\ver(u,v)) \sprod (\ver_u \times \ver_v) \dd A 
            = \int_0^1 \int_\frac{-\pi}{2}^\frac{\pi}{2} v \dd u \dd v
            = \cdots = \frac{\pi}{2} 
    \).
\end{ex}

\end{document}
