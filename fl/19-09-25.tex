\documentclass[a4paper]{article}
\usepackage{../flpack}

\begin{document}

\providecommand\fname{}
\renewcommand\fname{19-09-25}

\begin{defn}[Typ I-område]
    Ett område \(
        D \subseteq \realn^2
    \) är av \emph{Typ I} om det kan skrivas som \(
        D = \{ (x,y) \; | \; a \leq x \leq b, \; c(x) \leq y \leq d(x) \}
    \) där \(
        c(x)
    \) och \(
        d(x)
    \) är kontinuerliga.
\end{defn}

\begin{defn}[Typ II-område]
    Ett område \(
        D \subseteq \realn^2
    \) är av \emph{Typ II} om det kan skrivas som \(
        D = \{ (x,y) \; | \; a(y) \leq x \leq b(y), \; c \leq y \leq d \}
    \) där \(
        a(x)
    \) och \(
        b(x)
    \) är kontinuerliga.
\end{defn}

\begin{sats}
    Om \(
        D 
    \) är ett område av typ I eller II och \(
        f(x,y)
    \) är kontinuerlig på \(
        D
    \) är \(
        f
    \) integrerbar. 

    \begin{proof}
        Anta att \(
            D
        \) är ett område av typ I och att \(
            D \subseteq E = [a,b] \times [g,h]
        \) så att \(
            g \leq c(x) \; \forall \; x
        \) och \(
            h \geq d(x) \; \forall \; x
        \). Då är 
        \begin{align*}
            \iint_D f(x,y) \dd A &= \iint_R F(x,y) \dd A \\
                &= \int_a^b \int_g^h F(x,y) \dd y \dd x \\
                &= \{ F(x,y) = 0 \text{ om } g \leq y \leq c(x) \text{ el. } d(x) \leq y \leq h \} \\
                &= \int_a^b \int_{c(x)}^{d(x)} F(x,y) \dd y \dd x \\
                &= \{ F(x,y) = f(x,y) \text{ om } (x,y) \in D \} \\
                &= \int_a^b \int_{c(x)}^{d(x)} f(x,y) \dd y \dd x.
        \end{align*}

        Samma härledning gäller för typ II-områden.
    \end{proof}
\end{sats}

\end{document} 