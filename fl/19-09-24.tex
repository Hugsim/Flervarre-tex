\documentclass[a4paper]{article}
\usepackage{../flpack}

\begin{document}

\providecommand\fname{}
\renewcommand\fname{19-09-24}

\begin{sats}[Volym under graf]
    Anta att \(
        f
    \) är en funktion definierad på en mängd \(
        D = [a,b] \times [c\times d] = \{ (x,y) \; | \; a \leq x \leq b, \; c \leq y \leq d \}
    \).

    Volymen under grafen \(
        z = f(x,y)
    \) är då \(
        \iint_D f(x,y) \dd A
    \).

    \begin{proof}
        Dela in \(
            D
        \) i \(
            m \times n
        \) stycken lika stora rektanglar \(
            A_{ij}
        \) med area \(
            \Delta A = \frac{b-a}{m} \frac{d-c}{n}
        \) och låt punkten \(
            (x_{ij}, y_{ij}) \in A_{ij}
        \). Då är volymen under grafen ungefär \(
            \sum_{i=1}^m \sum_{j=1}^n f(x_{ij}, y_{ij}) \Delta A
        \).

        \begin{defn}[Integrerbarhet]
            Vi låter \(
                \iint_D f(x,y) \dd A = \lim_{m,n \to \infty} \sum_{i=1}^m \sum_{j=1}^n f(x_{ij}, y_{ij}) \Delta A
            \) om gränsvärdet existerar. Då säger vi att \(
                f
            \) är integrerbar.
        \end{defn}

        Om \(
            f(x,y) \geq 0
        \) och \(
            f
        \) är integrerbar är volymen mellan \(
            D
        \) och \(
            z = f(x,y)
        \) då \(
            \iint_D f(x,y) \dd A
        \).
    \end{proof}
\end{sats}

\begin{sats}[Krav för integrerbarhet]
    Om en funktion \(
        f
    \) är kontinuerlig på en mängd \(
        D
    \) är \(
        f
    \) integrerbar på \(
        D
    \).
\end{sats}

\begin{sats}[Fubinis sats]
    Om \(
        f(x,y)
    \) är kontinuerlig på \(
        D = [a,b] \times [c, d]
    \) är \[
        \iint_D f(x,y) \dd A = \int_a^b \int_c^d f(x,y) \dd y \dd x 
            = \int_c^d \int_a^b f(x,y) \dd x \dd y.
    \]

    \begin{proof}[Idé]
        Idén bakom satsen är att i två variabler är volymen integralen
        av tvärsnittsarean, likt att arean är integralen av höjden 
        i en variabel. Integralen av tvärsnittsarean borde bli
        samma oberoende av vilket håll vi tar tvärsnittsarean.
    \end{proof}
\end{sats}

\begin{defn}[Dubbelintegral över begränsade områden]
    Låt \(
        f(x,y)
    \) vara definierad på ett begränsat område \(
        D \subseteq \realn^2
    \). Låt också \(
        R
    \) vara en rektangel som helt innehåller \(
        D
    \) och definiera \(
        F(x,y) = \left\{ 
        \begin{matrix}
            f(x,y) \text{ om } (x,y) \in D \\ 
            0 \text{ annars}
        \end{matrix}\right.
    \). Då definierar man \(
        \iint_D f(x,y) \dd A = \iint_R F(x,y) \dd A
    \) om \(
        F
    \) är integrerbar.

    Eftersom \(
        F(x,y) = f(x,y)
    \) på hela \(
        D
    \) och \(
        F(x,y) = 0
    \) utanför \(
        D
    \) så borde volymen mellan \(
        z = f(x,y)
    \) och \(
        D
    \) vara samma som volymen mellan \(
        z = F(x,y)
    \) och \(
        R
    \) och bör därför vara en rimlig definition av en integral av \(
        f
    \) över \(
        D
    \).
\end{defn}

\end{document} 